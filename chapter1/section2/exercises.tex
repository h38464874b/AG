\documentclass{article}
\usepackage[margin=0.75in]{geometry}
\usepackage{amsmath}
\usepackage{amsthm}
\usepackage{amssymb}
\usepackage{enumitem}
\usepackage{tikz-cd}
\usepackage{yfonts}
\usepackage{mathrsfs}
\usepackage{xcolor}
\usepackage{physics}

\DeclareMathAlphabet{\mathpzc}{OT1}{pzc}{m}{it}

\newcommand{\goth}[1]{\mathfrak{#1}}
\newcommand{\reduced}[1]{#1_{\text{red}}}
\newcommand{\fF}{\mathscr{F}}
\newcommand{\fG}{\mathscr{G}}
\newcommand{\fE}{\mathscr{E}}
\newcommand{\fO}{\mathscr{O}}
\newcommand{\fL}{\mathscr{L}}
\newcommand{\fM}{\mathscr{M}}
\newcommand{\fI}{\mathscr{I}}
\newcommand{\fT}{\mathscr{T}}
\newcommand{\fK}{\mathscr{K}}
\newcommand{\fS}{\mathscr{S}}
\newcommand{\fN}{\mathscr{N}}
\newcommand{\fJ}{\mathscr{J}}
\newcommand{\fR}{\mathscr{R}}
\newcommand{\fH}{\mathscr{H}}
\newcommand{\PP}{\mathbb{P}}
\newcommand{\gm}{\goth{m}}
\newcommand{\A}{\mathbb{A}}
\newcommand{\R}{\mathbb{R}}
\newcommand{\C}{\mathbb{C}}
\newcommand{\Q}{\mathbb{Q}}
\newcommand{\N}{\mathbb{N}}
\newcommand{\Z}{\mathbb{Z}}
\newcommand{\G}{\mathbb{G}}
\newcommand{\gF}{\goth{F}}
\newcommand\srestr[2]{{\left.\kern-\nulldelimiterspace #1\vphantom{\small|} \right|_{#2}}}
\newcommand\restr[2]{{\left.\kern-\nulldelimiterspace #1 \vphantom{\big|} \right|_{#2}}}

\newtheorem{theorem}{Theorem}
\newtheorem{lemma}{Lemma}
\newtheorem{corollary}{Corollary}

\DeclareMathOperator{\id}{id}
\DeclareMathOperator{\bProj}{\mathpzc{Proj}}
\DeclareMathOperator{\Frac}{Frac}
\DeclareMathOperator{\rk}{rank}
\DeclareMathOperator{\pic}{Pic}
\DeclareMathOperator{\cacl}{CaCl}
\DeclareMathOperator{\trd}{tr.d.}
\DeclareMathOperator{\cl}{Cl}
\DeclareMathOperator{\depth}{depth}
\DeclareMathOperator{\codim}{codim}
\DeclareMathOperator{\Div}{Div}
\DeclareMathOperator{\coker}{coker}
\DeclareMathOperator{\len}{length}
\DeclareMathOperator{\height}{height}
\DeclareMathOperator{\supp}{Supp}
\DeclareMathOperator{\proj}{Proj}
\DeclareMathOperator{\im}{im}
\DeclareMathOperator{\Hom}{Hom}
\DeclareMathOperator{\Der}{Der}
\DeclareMathOperator{\spec}{Spec}

\author{James Lee}
\pagecolor[RGB]{20,20,20}
\color[RGB]{255,255,255}

\title{Chapter 1, Section 2}

\begin{document}
\maketitle
\begin{enumerate} [label=\textbf{\arabic*.}, leftmargin=0cm]

\item[\textbf{1.}] Prove the ``homogenous Nullstellensatz,'' which says if $\goth{a} \subseteq S$ is a homogenous ideal, and if $f \in S$ is a homogenous polynomial with $\deg{f} > 0$, such that $f(P) = 0$ for all $P \in Z(\goth{a})$ in $\PP^n$, then $f^q \in \goth{a}$ for some $q > 0$.

\begin{proof}
    We can apply the usual Nullstellensatz by viewing $f$ as a polynomial over $\A^{n + 1}$ and $Z(\goth{a})$ as the set $Z(\goth{a})'$ consisting of all $(n + 1)$-uples $Q = (a_0, \dots, a_{n + 1}) \in \A^{n + 1}$ such that $Q$ is the homogenous coordinate of some point $P \in Z(\goth{a})$.
    Since $f$ is homogenous, $f(P) = 0$ for $P \in \PP^n$ if and only if $f(Q) = 0$ where $P$ has $Q$ as homogenous coordinates.
    Thus, $f^q \in \goth{a}$ by the Nullstellensatz.
\end{proof}

\item[\textbf{2.}] For a homogenous ideal $\goth{a} \subseteq S$, show that the following conditions are equivalent:
\begin{itemize}
    \item[(i)] $Z(\goth{a}) = \emptyset$;

    \item[(ii)] $\sqrt{\goth{a}} =$ either $S$ or the ideal $S_+ = \bigoplus_{d > 0} S_d$;

    \item[(iii)] $\goth{a} \supseteq S_d$ for some $d > 0$.
\end{itemize}

\begin{proof}
    (i) $\implies$ (ii) Let $S = k[x_0, \dots, x_n]$.
    If $Z(\goth{a}) \empty$, then all homogenous polynomials in $S$ with degree $> 0$ are in $\sqrt{\goth{a}}$ by the Nullstellensatz, so we have $S_+ \subseteq \sqrt{\goth{a}}$.
    If $\sqrt{\goth{a}}$ contains any element in $S - S_+$, then $\sqrt{\goth{a}}$ contains an element in $S_0$, which is a unit.
    Hence, $\sqrt{\goth{a}} = S$.

    (ii) $\implies$ (iii) Either case $S_+ \subseteq \sqrt{\goth{a}}$, so $x_i^{d_i} \in \goth{a}$ for some $d_i \geq 0$ for each $i$, which implies $S_d \subseteq \goth{a}$ where $d = \max{d_i}$.

    (iii) $\implies$ (i) Let $d > 0$ be the smallest integer such that $\goth{a} \supseteq S_d$, then we have $\goth{a} \supseteq \bigoplus_{l \geq d} S_l$ so that $x_i^l \in \goth{a}$ for all $0 \leq i \leq n$ and $l \geq d$.
    If there exists $P = (a_0, \dots, a_n) \in Z(\goth{a})$, then $a_i^d = a_i^{d + 1} = 0$, which implies $a_i = 0$ for all $0 \leq i \leq n$, which is impossible.
\end{proof}

\item[\textbf{4.}] \begin{itemize}
    \item[(a)] There is a $1$-$1$ inclusion-reversing correspondence between algebraic sets in $\PP^n$, and homogenous radicals of $S$ not equal $S_+$, given by $Y \mapsto I(Y)$ and $\goth{a} \mapsto Z(\goth{a})$.
    \textit{Note:} Since $S_+$ does not occur in this correspondence, it is sometimes called the \textit{irrelevant} maximal ideal of $S$.

    \item[(b)] An algebraic set $Y \subseteq \PP^n$ is irreducible if and only if $I(Y)$ is a prime ideal.

    \item[(c)] Show that $\PP^n$ itself is irreducible.
\end{itemize}

\begin{proof} $ $ \vspace{0pt}
    \begin{itemize} [leftmargin=0cm]
        \item[(a)] Only the last part that states $S_+$ does not occur in this correspondence is new. Indeed, $I(Y) \supseteq S_+$ for some algebraic set $Y = Z(\goth{a})$, then $Z(\goth{a}) = \emptyset$ b (I, Ex. 2.2), and any constant polynomial $f \in S_0$ vacuously satisfies $f(P) = 0$ for all $P \in Z(\goth{a})$. Hence, $I(Y) = S$.

        \item[(b)] By the $1$-$1$ correspondence from (a), $I(Y)$ is a homogenous ideal, so it is sufficient to show for any two homogenous elements $f$, $g$, that $fg \in I(Y)$ implies $f \in I(Y)$ or $g(Y)$. Indeed, if $fg \in I(Y)$, then $Y \subseteq Z(fg) = Z(f) \cup Z(g)$, thus $Y = (Y \cap Z(f)) \cup (Y \cap Z(g))$ both being closed subsets of $Y$. Since $Y$ is irreducible, we have either $Y = Y \cap Z(f)$, in which case $Y \subseteq Z(f)$, or $Y \subseteq Z(g)$. Hence, either $f \in I(Y)$ or $g \in I(Y)$.

        Conversely, let $\goth{p}$ be a homogenous prime ideal, and suppose that $Z(\goth{p}) = Y_1 \cup Y_2$. Then $\goth{p} = I(Y_1) \cap I(Y_2)$, so either $\goth{p} = I(Y_1)$ or $\goth{p} = I(Y_2)$. Thus, $Z(\goth{p}) = Y_1$ or $Y_2$, hence it is irreducible.

        \item[(c)] $I(\PP^n) = (0)$, which is a prime ideal.
    \end{itemize}
\end{proof}

\item[\textbf{5.}] \begin{itemize}
    \item[(a)] $\PP^n$ is a noetherian topological space.
    \item[(b)] Every algebraic set in $\PP^n$ can be written uniquely as a finite union of irreducible algebraic sets, no one containing another. These are called its \textit{irreducible components}.
\end{itemize}

\begin{proof} $ $ \vspace{0pt}
    \begin{itemize} [leftmargin=0cm]
        \item[(a)] $\PP^n$ is covered by the open set $U_i$ defined the non-vanshing of the $i$th homogenous coordinate, each of which is homeomorphic to the affine plane $\A^n$.
        The affine plane is a noetherian topological space, so it suffices to show a finite union of noetherian topological spaces is also noetherian.
        An equivalent condition for a space to be noetherian is if every subset is quasi-compact by (A.M. p. 79). If $Y$ is any subset of $\PP^n$, then $Y = Y \cap \bigcup U_i$, each of which is quasi-compact in the induced topology of $U_i$, so $Y$ is a finite union of quasi-compact sets. Hence, $Y$ is quasi-compact.

        \item[(b)] Follows from (a), (1.5), and (I, Ex. 2.4b).
    \end{itemize}
\end{proof}

\item[\textbf{6.}] If $Y$ is a projective variety with homogenous coordinate ring $S(Y)$, show that $\dim{S(Y)} = \dim{Y} + 1$.

\begin{proof}
    Let $\varphi_i : U_i \to \A^n$ be the homeomoprhism of (2.2),  let $Y_i$ be the affine variety $\varphi_i(Y \cap U_i)$, and let $A(Y_i)$ be its affine coordinate ring. If $g(y_1, \dots, y_n)$ is an element of $A(Y_i)$, then define the map $A(Y_i) \rightarrow S(Y)_{x_i}$ as $g(x_0 / x_i, \dots, x_n / x_i)$, or equivalently $g \mapsto \varphi_i^* g = g \circ \varphi_i$; thus we can identify $A(Y_i)$ with the subring of elements of degree $0$ of the localized ring $S(Y)_{x_i}$. Then, $S(Y)_{x_i} \simeq A(Y_i)[x_i, x_i^{-1}]$ since every monomial in $k[x_0, \dots, x_n]$ can be written as
    \begin{equation*}
        x_0^{d_0} \cdots x_i^{d_i} \cdots x_n^{d_n} = \frac{x_0^{d_0} \cdots \widehat{x_i^{d_i}} \cdots x_n^{d_n}}{x_i^{d_0 + \cdots + \widehat{d_i} + \cdots + d_n}} x_i^{d_0 + \cdots + d_n},
    \end{equation*}
    where $~\widehat{}~$ denotes omission, and the quotient is in the image of $A(Y_i) \to S(Y)_{x_i}$. By (1.8A), the dimension of $S(Y)$ is equal to the transcendence degree of the quotient field of $S(Y)$, which is isomorphic to the quotient field of $S(Y)_{x_i}$. The dimension of $A(Y_i)$ is equal to the dimension to $Y_i$ by (1.7). Therefore, we have
    \begin{align*}
        \dim{S(Y)} & = \dim{S(Y)_{x_i}} \\
        & = \dim{A(Y_i)[x_i, x_i^{-1}]} \\
        & = \dim{A(Y_i)} + 1 \\
        & = \dim{Y_i} + 1.
    \end{align*}
    Since the $Y_i$ cover $Y$, $\dim{Y} = \sup{\dim{Y_i}}$, so $\dim{S(Y)} = \dim{Y} + 1$ By (I, Ex. 1.10).
\end{proof}

\item[\textbf{7.}] \begin{itemize}
    \item[(a)] $\dim{\PP^n} = n$.
    \item[(b)] If $Y \subseteq \PP^n$ is quasi-projective variety, then $\dim{Y} = \dim{\overline{Y}}$.
\end{itemize}

\begin{proof} $ $ \vspace{0pt}
   \begin{itemize}
        \item[(a)] $\dim{\PP^n} = \dim{k[x_0, \dots, x_n]} - 1 = n$.
        \item[(b)] By Exercise 2.6, we have $\dim{Y} = \dim{Y_i}$ if $Y_i$ is non-empty, so by (1.10) $\dim{\overline{Y}} = \dim{\overline{Y}_i} = \dim{Y_i} = \dim{Y}$.
   \end{itemize} 
\end{proof}

\item[\textbf{8.}] A projective variety $Y \subseteq \PP^n$ has dimension $n - 1$ if and only if it is the zero set of a single irreducible homogenous polynomial $f$ of positive degree. $Y$ is called a \textit{hypersurface} in $\PP^n$.

\begin{proof}
    If $Y$ has dimension $n - 1$, then by (1.13) $Y$ is the union of affine varieties $Y_i$ of dimension $n - 1$, so by the map $\beta$ in the proof of (2.2) and (1.13) each $Y_i$ is the zero set of an irreducible homogenous polynomial $f$ of positive degree, hence $Y = \bigcup Y_i = \bigcup_{Y_i \neq \emptyset} Z(f_i) = Z(f_0 \cdots f_n)$. Conversely, if $Y = Z(f)$ for some homogenous polynomial $f$ of positive degree, then $Y_i = Z(\alpha(f))$ where $\alpha$ is the map defined in the proof of (2.2), hence by (1.13) and Exercise 6 we have $\dim{Y} = \dim{Y_i} = n - 1$.
\end{proof}

\item[\textbf{12.}] \textit{The $d$-Uple Embedding.}
For given $n, d > 0$, let $M_0, M_1, \dots, M_N$ be all the monomials of degree $d$ in the $n + 1$ variables $x_0, \dots, x_n$, where $N = {n + d \choose n} - 1$.
We define a mapping $\rho_d : \PP^n \to \PP^N$ by sending the point $P = (a_0, \dots, a_n)$ to the point $\rho_d(P) = (M_0(a), \dots, M_N(a))$ obtained by substituting the $a_i$ in the monomials $M_j$.
This is called the $d$-uple \textit{embedding} of $\PP^n$ in $\PP^N$.
For example, if $n = 1$, $d = 2$, then $N = 2$, and the image of $Y$ of the $2$-uple embedding of $\PP^1$ in $\PP^2$ is a conic.
\begin{itemize}
    \item[(a)] Let $\theta : k[y_0, \dots, y_N] \to k[x_0, \dots, x_n]$ be the homomorphism defined by sending $y_i$ to $M_i$, and let $\goth{a}$ be the kernel of $\theta$. Then $\goth{a}$ is a homogenous prime ideal, and so $Z(\goth{a})$ is a projective variety in $\PP^N$.
    \item[(b)] Show that the image of $\rho_d$ is exactly $Z(\goth{a})$.
    \item[(c)] Now show that $\rho_d$ is a homeomoprhism of $\PP^n$ onto the projective varieety $Z(\goth{a})$.
    \item[(d)] Show that the twisted cubic curve in $\PP^3$ (Ex. 2.9) is equal to the $3$-uple embedding of $\PP^1$ in $\PP^3$, for suitable choice of coordinates.
\end{itemize}

\begin{proof} $ $ \vspace{0pt}
    \begin{itemize} [leftmargin=0cm]
        \item[(a)] The image of $\theta$ is an integral domain, so $\goth{a}$ is a prime ideal, and it is clearly homogenous since each $y_i$ is sent to a polynomial of same degree.

        \item[(b)] If $Q \in \im{\rho_d}$, then $Q = \rho_d(P)$ for some $P \in \PP^n$, for any $f \in \goth{a}$ we have
        \begin{equation*}
            f(Q) = f(\rho_d(P)) = \theta(f)(P) = 0 \implies Q \in Z(\goth{a}).
        \end{equation*}
        Before proving the converse direction, consider the case when $n = 1$ and $d = 2$, so that $\rho_d : \PP^1 \to \PP^2$ is defined as $(b_0, b_1) \mapsto (b_0^2, b_0 b_1, b_1^2)$. Then, the polynomial $y_0 y_2 - y_1^2$ is in $\goth{a}$. If $Q = (a_0, a_1, a_2) \in Z(\goth{a})$, then since $k$ is algebraically closed we have
        \begin{equation*}
            a_0 a_2 - a_1^2 = 0 \implies a_1 = \pm \sqrt{a_0 a_2} \implies \rho_d(\sqrt{a_0}, \sqrt{a_2}) = (a_0, a_1, a_2).
        \end{equation*}
        Returning to the general case, if $Q = (a_0, \dots, a_N) \in Z(\goth{a})$, indexing $M_i$ using the stars and bars method, we have $\rho_d(\sqrt[d]{a_0}, \sqrt[d]{a_d}, \dots, \sqrt[d]{a_N}) = Q$.

        \item[(c)] $\rho_d$ is clearly bijective, so it will be sufficient to show that the closed sets of $\PP^n$ are identified with the closed sets of $Z(\goth{a})$ by $\rho_d$. Let $Y \subseteq \PP^n$ be a closed subset, so $Y = Z(T)$ for some subset $T \subseteq k[x_0, \dots, x_n]$, then it is easy to see that $\rho_d(Y) = Z(\theta^{-1}(T)) \cap Z(\goth{a})$. Conversely, let $W$ be a closed subset of $Z(\goth{a})$. Let $\overline{W}$ be its closure in $\PP^N$. This is an algebraic set, so $\overline{W} = Z(T')$ for some $T' \subseteq k[y_0, \dots, y_N]$, hence $\rho_d^{-1}(W) = \rho_d^{-1}(\overline{W}) = Z(\theta(T'))$.

        \item[(d)] Let $(S, T)$ and $(X, Y, Z, W)$ be homogenous coordinates of $\PP^1$ and $\PP^3$, respectively. Then, the $3$-uple embedding of $\PP^1$ in $\PP^3$ is given by
        \begin{equation*}
            (S, T) \mapsto (S^3, S^2 T, ST^2, T^3).
        \end{equation*}
        Let $Y$ be the twisted cubic curve in $\A^3$ and let $\overline{Y}$ be its projective closure in $\PP^3$, then we have
        \begin{equation*}
            \varphi_3^{-1}(Y) = \left\{ \bigg( \frac{u}{v}, \frac{u^2}{v^2}, \frac{u^3}{v^3}, 1 \bigg) \bigg| u, v \in k, ~ v \neq 0 \right\} \implies \overline{Y} = \{ (u^3, u^2v, uv^2, v^3) \mid u, v \in k \}.
        \end{equation*}
    \end{itemize}
\end{proof}


\item[\textbf{14.}] \textit{Segre Embedding.} Let $\psi : \PP^r \times \PP^s \to \PP^N$ be the map defined by sending the ordered pair $(a_0, \dots, a_r) \times (b_0, \dots, b_s)$ to $(\dots, a_i b_j, \dots)$ in lexicographic order, where $N = rs + r + s$. Note that $\psi$ is well-defined and injective. It is called the \textit{Segre embedding}. Show that the image of $\psi$ is a \textit{subvariety} of $\PP^N$.

\begin{proof}
    Let the homogenous coordinates of $\PP^N$ be $\{ z_{ij} \mid i = 0, \dots, r, j = 0, \dots, s \}$, and let $\goth{a}$ be the kernel of the homomorphism $\theta : k[\{ z_{ij} \}] \to k[x_0, \dots, x_r, y_0, \dots, y_s]$ which sends $z_{ij}$ to $x_i y_j$. If $P \in \im{\psi}$, then $P = \psi(Q, R)$ for some $Q \in \PP^r$ and $R \in \PP^s$, then for any $f \in \goth{a}$ we have
    \begin{equation*}
        f(P) = f(\psi(Q, R)) = \theta(f)(Q, R) = 0 \implies P \in Z(\goth{a}).
    \end{equation*}
    Conversely, viewing points of $\PP^N$ as $(r + 1)\times(s + 1)$-matrices, the variety $Z(\goth{a})$ is defined as the vanshing of all $2\times2$-minors, i.e. $z_{ij}z_{kl} = z_{il}z_{jk}$ for all $0 \leq i, k \leq r$ and $0 \leq j, l \leq s$. This means $\{ z_{ij} \} \in Z(\goth{a})$ has rank $1$, so it can be expressed as the outer product of two vectors in $k^{r + 1}$ and $k^{s + 1}$, which is exactly the mapping defined by $\psi$.
\end{proof}

\item[\textbf{15.}] \textit{Quadric Surface in $\PP^3$.} Consider the surface $Q$ (a \textit{surface} is a variety of dimension $2$) in $\PP^3$ defined by the equation $xy - zw = 0$. 
\begin{itemize}
    \item[(a)] Show that $Q$ is equal to the Segre embedding of $\PP^1 \times \PP^1$ in $\PP^3$, for suitable choice of coordinates.
    \item[(b)] Show that $Q$ contains two families of lines (a \textit{line} is a linear variety of dimension $1$) $\{ L_t \}$, $\{ M_t \}$, each parametrized by $t \in \PP^1$, with the proeprties that if $L_t \neq L_u$, then $L_t \cap L_u = \emptyset$; if $M_t \neq M_u$, $M_t \cap M_u = \emptyset$, and for all $t$, $u$, $L_t \cap M_u = \text{one point}$.
    \item[(c)] Show that $Q$ contains other curves besides these lines, and deduce that the Zariski topology on $Q$ is not homeomoprhic via $\psi$ to the product topology on $\PP^1 \times \PP^1$ (where each $\PP^1$ has its Zariski topology).
\end{itemize}

\begin{proof} $ $ \vspace{0pt}
    \begin{itemize} [leftmargin=0cm]
        \item[(a)] The kernel of the mapping $k[z_{00}, z_{01}, z_{10}, z_{11}] \to k[x_0, x_1, y_0, y_1]$ as in Exercise 14 is generated by $z_{00} z_{11} - z_{01} z_{10}$. Then $Z(z_{00} z_{11} - z_{01} z_{11})$ is equal to $Q$ for a suitable choice of coordinates.

        \item[(b)] From here, we assume $Q$ is defined by $xw - yz = 0$ (the author was too lazy to fix his mistake after realizing it at (c)). Consider the Segre embedding $\psi : \PP^1 \times \PP^1 \to \PP^3$ defined by
        \begin{equation*}
            (t_0, t_1) \times (u_0, u_1) \mapsto (t_0 u_0, t_0 u_1, t_1 u_0, t_1 u_1).
        \end{equation*}
        Fixing the first entry as $t = (t_0, t_1) \in \PP^1$, we obtain an embedding of $\PP^1$ in $\PP^3$, which can be identified as the intersection of the zero set of the following linear polynomials
        \begin{equation*}
            t_1 x - t_0 z = 0, \quad t_1 y - t_0 w = 0,
        \end{equation*}
        so we have one family of lines $\{ L_t \}$, and we can obtain a second family of lines in the same manner by fixing the second entry of the map $\psi$, where the line $M_u$ for $u = (u_0, u_1) \in \PP^1$ is defined by the intersection of the zero set of the linear polynomials
        \begin{equation*}
            u_1 x - u_0 y = 0, \quad u_1 z - u_0 w = 0.
        \end{equation*}
        Then $L_t \neq L_u$ implies $L_t \cap L_u = \emptyset$ follows from the fact that $\psi$ is an embedding, in particular $\psi$ is injective. To show the intersection of $L_t$ and $M_u$ is a single point for any $t, u \in \PP^1$, it suffices to show the linear polynomials defined above intersect at exactly one point. Without loss of generality assume $t_1, u_1 \neq 0$, then setting $\lambda = t_0 / t_1$ and $\mu = u_0 / u_1$, we have
        \begin{equation*}
            x = \lambda z, \quad y = \lambda w, \quad x = \mu y, \quad z = \mu w \implies L_t \cap M_u = \{ (\lambda \mu, \lambda, \mu, 1) \}.
        \end{equation*}

        \item[(c)] Consider the curve $K$ in $Q$ defined by $x = w$. It is clearly not a line since it is the intersection of a nonlinear curve and a linear curve; however it is a closed subset of $Q$. On the other hand, $\psi^{-1}(K)$ is the diagonal in $\PP^1 \times \PP^1$, which is certainly not closed since $\PP^1$ is not Hausdorff.
    \end{itemize} 
\end{proof}

\end{enumerate}

\end{document}

% \item \begin{itemize}
%     \item[(a)] The intersection of two varities need not be a variety. For example, let $Q_1$ and $Q_2$ be the quadric surfaces in $\PP^3$ given by the equations $x^2 - yw = 0$ and $xy - zw = 0$, respectively. Show that $Q_1 \cap Q_2$ is the union of a twisted cubic curve and a line.
%     \item[(b)] Even if the intersection of two varieties is a variety, the ideal of the intersection may not be the sum of the ideals. For example, let $C$ be the conic in $\PP^2$ given by the equations $x^2 - yz = 0$. Let $L$ be the line given by $y = 0$. Show that $C \cap L$ consists of one point $P$, but that $I(C) + I(L) \neq I(P)$.
% \end{itemize}

% \item \textit{Complete intersections.} A variety $Y$ of dimension $r$ in $\PP^n$ is a \textit{(strict) complete intersection} if $I(Y)$ can be generated by $n - r$ elements. $Y$ is a \textit{set-theoretic complete intersection} if $Y$ can be written as the intersection of $n - r$ hypersurfaces.
% \begin{itemize}
%     \item[(a)] Let $Y$ be a variety in $\PP^n$, let $Y = Z(\goth{a})$; and suppose that $\goth{a}$ can be generated by $q$ elements. Then show that $\dim{Y} \geq n - q$.
%     \item[(b)] Show that a strict complete intersection is a set-theoretic complete intersection.
%     \item[(c)] The converse of (b) is false. For example let $Y$ be the twisted cubnic curve in $\PP^3$ (Ex. 2.9). Show that $I(Y)$ cannot be generated by two elements. On the other hand, find hypersurfaces $H_1$, $H_2$ of degrees $2$, $3$ respectively, such that $Y = H_1 \cap H_2$.
%     \item[(d)] It is an unsolved problem whether every closed irreducible curve in $\PP^3$ is a set-theoretic intersection of two surfaces.
% \end{itemize}

% \item \begin{itemize}
%     \item[(a)] If $T_1 \subseteq T_2$ are subsets of $S^h$, then $Z(T_1) \supseteq Z(T_2)$.
%     \item[(b)] If $Y_1 \subseteq Y_2$ are subsets of $\PP^n$, then $I(Y_1) \supseteq I(Y_2)$.
%     \item[(c)] For any two subsets $Y_1$, $Y_2$ of $\PP^n$, $I(Y_1 \cup Y_2) = I(Y_1) \cap I(Y_2)$.
%     \item[(d)] If $\goth{a} \subseteq S$ is a homogenous ideal with $Z(\goth{a}) \neq \emptyset$, then $I(Z(\goth{a})) = \sqrt{\goth{a}}$.
%     \item[(e)] For any subset $Y \subseteq \PP^n$, $Z(I(Y)) = \overline{Y}$.
% \end{itemize}

% \item \textit{Projective Closure of an Affine Variety.} If $Y \subseteq \A^n$ is an affine variety, we identify $\A^n$ with an open set $U_0 \subseteq \PP^n$ by the homeomorphism $\varphi_0$. Then we can speak of $\overline{Y}$, the closure of $Y$ in $\PP^n$, which is called the \textit{projective closure} of $Y$.
% \begin{itemize}
%     \item[(a)] Show that $I(\overline{Y})$ is the ideal generated by $\beta(I(Y))$, using the notation of the proof of (2.2).
%     \item[(b)] Let $Y \subseteq \A^3$ be the twisted cubic of (Ex. 1.2). Its projective closure $\overline{Y} \subseteq \PP^3$ is called the \textit{twisted cubic curve} in $\PP^3$. Find generators for $I(Y)$ and $I(\overline{Y})$, and use this example to show that if $f_1, \dots, f_r$ generated $I(Y)$, then $\beta(f_1), \dots, \beta(f_r)$ do \textit{not} necessarily generated $I(\overline{Y})$.
% \end{itemize}

% \item \textit{The Cone Over a Projective Variety.} Let $Y \subseteq \PP^n$ be a nonempty algebraic set, and let $\theta : \A^{n + 1} - \{(0, \dots, 0) \} \to \PP^n$ be the map which sends the point with affine coordinates $(a_0, \dots, a_n)$ to the point with homogenous coordinates $(a_0, \dots, a_n)$. We define the \textit{affine cone} over $Y$ to be
% \begin{equation*}
%     C(Y) = \theta^{-1}(Y) \cup \{ (0, \dots, 0) \}.
% \end{equation*}
% \begin{itemize}
%     \item[(a)] Show that $C(Y)$ is an algebraic set in $\A^{n + 1}$, whose ideal is equal to $I(Y)$ considered as an ordinary ideal in $k[x_0, \dots, x_n]$.
%     \item[(b)] $C(Y)$ is irreducible if and only if $Y$ is.
%     \item[(c)] $\dim{C(Y)} = \dim{Y} + 1$.
% \end{itemize}
% Sometimes we consider the projective closure $\overline{C(Y)}$ of $C(Y)$ in $\PP^{n + 1}$. This is called the \textit{projective cone} over $Y$.
% \begin{figure}[H]
%     \centering
%     \includegraphics*[width=9cm]{Screenshot 2024-07-27 at 4.33.42 PM.png}
%     \caption{The cone over a curve in $\PP^2$.}
% \end{figure}

% \item \textit{Linear Varieties in $\PP^n$.} A hypersurface defined by a linear polynomial in is called a \textit{hyperplane}.
% \begin{itemize}
%     \item[(a)] Show that the following two conditions are equivalent for a variety $Y$ in $\PP^n$:
%     \begin{itemize}
%         \item[(i)] $I(Y)$ can be generated by linear polynomials.
%         \item[(ii)] $Y$ can be written as an intersection of hyperplanes.
%     \end{itemize}
%     In this case we say that $Y$ is a \textit{linear variety} in $\PP^n$.
%     \item[(b)] If $Y$ is a linear variety of dimension $r$ in $\PP^n$, show that $I(Y)$ is minimally generated by $n - r$ linear polynomials.
%     \item[(c)] Let $Y$, $Z$ be linear varieties in $\PP^n$, with $\dim{Y} = r$, $\dim{Z} = s$. If $r + s - n \geq 0$, then $Y \cap Z$ is a linear variety of dimension $\geq r + s - n$.
% \end{itemize}

% \item Let $Y$ be the image of the $2$-uple embedding of $\PP^2$ in $\PP^5$. This is the \textit{Veronese surface}. If $Z \subseteq Y$ is a closed curve (a \textit{curve} is a variety of dimension $1$), show that there exists a hypersurface $V \subseteq \PP^5$ such that $V \cap Y = Z$.
