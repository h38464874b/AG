\documentclass{article}
\usepackage[margin=0.75in]{geometry}
\usepackage{amsmath}
\usepackage{amsthm}
\usepackage{amssymb}
\usepackage{enumitem}
\usepackage{tikz-cd}
\usepackage{yfonts}
\usepackage{mathrsfs}
\usepackage{xcolor}
\usepackage{physics}

\DeclareMathAlphabet{\mathpzc}{OT1}{pzc}{m}{it}

\newcommand{\goth}[1]{\mathfrak{#1}}
\newcommand{\reduced}[1]{#1_{\text{red}}}
\newcommand{\fF}{\mathscr{F}}
\newcommand{\fG}{\mathscr{G}}
\newcommand{\fE}{\mathscr{E}}
\newcommand{\fO}{\mathscr{O}}
\newcommand{\fL}{\mathscr{L}}
\newcommand{\fM}{\mathscr{M}}
\newcommand{\fI}{\mathscr{I}}
\newcommand{\fT}{\mathscr{T}}
\newcommand{\fK}{\mathscr{K}}
\newcommand{\fS}{\mathscr{S}}
\newcommand{\fN}{\mathscr{N}}
\newcommand{\fJ}{\mathscr{J}}
\newcommand{\fR}{\mathscr{R}}
\newcommand{\fH}{\mathscr{H}}
\newcommand{\PP}{\mathbb{P}}
\newcommand{\gm}{\goth{m}}
\newcommand{\A}{\mathbb{A}}
\newcommand{\R}{\mathbb{R}}
\newcommand{\C}{\mathbb{C}}
\newcommand{\Q}{\mathbb{Q}}
\newcommand{\N}{\mathbb{N}}
\newcommand{\Z}{\mathbb{Z}}
\newcommand{\G}{\mathbb{G}}
\newcommand{\gF}{\goth{F}}
\newcommand\srestr[2]{{\left.\kern-\nulldelimiterspace #1\vphantom{\small|} \right|_{#2}}}
\newcommand\restr[2]{{\left.\kern-\nulldelimiterspace #1 \vphantom{\big|} \right|_{#2}}}

\newtheorem{theorem}{Theorem}
\newtheorem{lemma}{Lemma}
\newtheorem{corollary}{Corollary}

\DeclareMathOperator{\id}{id}
\DeclareMathOperator{\bProj}{\mathpzc{Proj}}
\DeclareMathOperator{\Frac}{Frac}
\DeclareMathOperator{\rk}{rank}
\DeclareMathOperator{\pic}{Pic}
\DeclareMathOperator{\cacl}{CaCl}
\DeclareMathOperator{\trd}{tr.d.}
\DeclareMathOperator{\cl}{Cl}
\DeclareMathOperator{\depth}{depth}
\DeclareMathOperator{\codim}{codim}
\DeclareMathOperator{\Div}{Div}
\DeclareMathOperator{\coker}{coker}
\DeclareMathOperator{\len}{length}
\DeclareMathOperator{\height}{height}
\DeclareMathOperator{\supp}{Supp}
\DeclareMathOperator{\proj}{Proj}
\DeclareMathOperator{\im}{im}
\DeclareMathOperator{\Hom}{Hom}
\DeclareMathOperator{\Der}{Der}
\DeclareMathOperator{\spec}{Spec}

\author{James Lee}
\pagecolor[RGB]{20,20,20}
\color[RGB]{255,255,255}

\title{Chapter 1, Section 7}

\begin{document}
\maketitle
\begin{enumerate} [label=\textbf{\arabic*.}, leftmargin=0cm]

\item[\textbf{1.}] \begin{itemize}
  \item[(a)] Find the degree of the $d$-uple embedding of $\PP^n$ in $\PP^N$.
  \item[(b)] Find the degree of the Segre embedding of $\PP^r \times \PP^s$ in $\PP^N$.
\end{itemize}

\begin{proof} $ $ \vspace{0pt}
  \begin{itemize}
    \item[(a)] The homogenous coordinate ring of the $d$-uple embedding is the $k[y_0, \dots, y_n]$ where each $y_i$ has degree $d$ as a graded $k[x_0, \dots, x_n]$-module. Thus, the Hilbert polynomial of the $d$-uple embedding is
    \begin{equation*}
      P(z) = {n + dz \choose n} = \frac{(dz + n)(dz + n - 1) \cdots (dz)}{n!} = \frac{d^n}{n!}z^n + \cdots,
    \end{equation*}
    hence the degree is $d^n$.

    \item[(b)] We have seen in Exercise 2.14 that the homogenous coordinate ring of the Segre embedding is isomorphic to $k[\{x_i y_j \}]$, which is a subring of $k[x_0, \dots, x_r, y_0, \dots, y_s]$. The grading is given by $\bigoplus_{k = 0}^\infty M_k$ where $M_k$ is the set of polynomials of degree $2k$ in $k[\{ x_i y_i \}]$. Since each monomial is made up of half $x_i$'s and half $y_j$'s, the Hilbert polynomial is
    \begin{equation*}
      P(z) = {r + z \choose r} {s + z \choose s} = \frac{1}{r!s!}z^{r + s} + \cdots,
    \end{equation*}
    hence the degree is ${r + s \choose r}$.
  \end{itemize}
\end{proof}

\item[\textbf{3.}] \textit{Dual Curve.} Let $Y \subseteq \PP^2$ be a curve. We regard the set of lines in $\PP^2$ as another projective space, $(\PP^2)^*$, by taking $(a_0, a_1, a_2)$ as homogenous coordinates of the line $L : a_0 x_0 + a_1 x_1 + a_2 x_2 = 0$. For each nonsingular point $P \in Y$, show that there is a unique line $T_P(Y)$ whose intersection multiplicity with $Y$ at $P$ is $ > 1$. This is the \textit{tangent line} to $Y$ at $P$. Show that the mapping $P \mapsto T_P(Y)$ defines a \textit{morphism} of $\text{Reg}~Y$ (the set of nonsingular points of $Y$) into $(\PP^2)^*$. The closure of the image of this morphism is called the dual curve $Y^* \subseteq (\PP^2)^*$ of $Y$.

\begin{proof}
  Let $x_0, x_1, x_2$ be homogenous coordinates for $\PP^2$, and let $f$ be an irreducible homogenous polynomial in $S = k[x_0, x_1, x_2]$ that defines $Y$ and let $P = (c_0, c_1, c_2)$ be a nonsingular point on $Y$. Then $\PP^2$ can be covered by $U_i$ where $U_i \cong \A^2$ and is defined by $x_i \neq 0$. Since every $P$ is contained in some $U_i$, we find an affine line in $U_i$ that is tangent to the affine variety $Y_i := Y \cap U_i$. We claim that the projective closure of this line is the desired unique line $T_P(Y)$.

  Without loss of generality, assume $c_0 = 1$ so that $P \in U_0$. Identifying $U_0$ with $\A_k^2$ with coordinates $y_1$, $y_2$, we have $P_0 = (c_1, c_2) \in \A_k^2$ identified with $P$, and $Y_0$ is defined by the vanishing of $g(y_1, y_2) = f(1, y_1, y_2)$. Assume $c_1 = c_2 = 0$ for simplicity. Let $\goth{m}_{P_0}$ be the maximal ideal in $S(U_0) \simeq k[y_1, y_2]$ corresponding to the point $P_0$. Since $Y$ is nonsingular at $P$, $Y_0$ is nonsingular at $P_0$, which means $a_i = (\partial g / \partial y_i)(P_0)$ for $i = 1, 2$ are not both zero. In particular, we have $g \notin \goth{m}_{P_0}^2$, so we can write
  \begin{equation*}
    g = a_1 y_1 + a_2 y_2 + (\text{higher degree terms}),
  \end{equation*}
  thus any linear polynomial in $\goth{m}_P^2 + (g)$ must be a constant multiple of $t =  a_1 y_1 + a_2 y_2 = 0$. We define $T_P(Y)$ to be the projective closure of $t$. We show it is the desired unique line by showing if a line $H : h(x_0, x_1, x_2) = b_0 x_0 + b_1 x_1 + b_2 x_2 = 0$ has intersection multiplicity with $Y$ at $P$ $> 1$, then $H \cap U_0 = T$, which implies $T_P(Y) = H$ since $H \cap U_0 = T$ is dense in both $H$ and $T$. Setting $S(Y) = S/(f)$ and $\goth{m}_P$ as the homogenous prime ideal in $S$ corresponding to the point $P \in \PP^2$, we have $S / (f, h) = S(Y) / (h)$, thus
  \begin{align*}
    i(Y, H ; P) > 1 \quad & \iff \quad (S(Y) / (h))_{\goth{m}_P} \text{ has length $k > 1$ as an $S(Y)_{\goth{m}_P}$-module} \\
    & \iff \quad \goth{m}_P / (h) \text{ has length $k$ as an $S(Y)$-module} \\
    & \iff \quad (h) = \goth{m}_P^k \text{ as an extended ideal in $S(Y)_{\goth{m}_P}$ }
  \end{align*}
  since $S(Y)_{\goth{m}_P}$ is a discrete valuation ring by non-singularity of $Y$ at $P$. In terms of affine coordinates, let $\ell = b_0 + b_1 y_1 + b_2 y_2$ so that $H \cap U_0$ can be identified with the affine variety defined by the vanishing of $\ell$, where $b_0 = 0$ necessarily since $P_0 = (0, 0)$ is a root of $\ell$. Then, by the statements above, we have $\ell \in \goth{m}_{P_0}^k + (g) \subseteq \goth{m}_{P_0}^2 + (g)$, which is possible if and only if $\ell = \lambda t$ for some $\lambda \neq 0$, hence $H \cap U_0 = T$.
\end{proof}

\item[\textbf{4.}] Given a curve $Y$ of degree $d$ in $\PP^2$, show that there is a nonempty open subset $U$ of $(\PP^2)^*$ in its Zariski topology such that for each $L \in U$, $L$ meets $Y$ in exactly $d$ points. This result shows that we could have defined the degree of $Y$ to be the number $d$ such that almost all lines in $\PP^2$ meet $Y$ in $d$ points, where "almost all" refers to a nonempty open set of the set of lines, when this set is identified with the dual projective space $(\PP^2)^*$.

\begin{proof}
  Following the hint, we show that the set of lines in $(\PP^2)^*$ which are either tangent to $Y$ or pass through a singular point of $Y$ is contained in a property closed subset. By the previous exercise, a line can meet $Y$ at exactly $d$ points if and only if it has intersection multiplicity equal to $1$ at every point of intersection, thus the set of lines which are either tangent to $Y$ or pass through a singular point of $Y$ is contained in the dual curve $Y^*$ in $(\PP^2)^*$, hence the set of lines that meet $Y$ at exactly $d$ points is contained in the open subset $(\PP^2)^* - Y^*$.  
\end{proof}

\end{enumerate}

\end{document}

% \item Let $Y$ be a variety of dimension $r$ in $\PP^n$, with Hilbert polynomial $P_Y$. We define the \textit{arithmetic genus} of $Y$ to be $p_a(Y) = (-1)^r(P_Y(0) - 1)$. This is an important invariant which is independent of the projective embedding of $Y$.
% \begin{itemize}
%   \item[(a)] Show that $p_a(\PP^n) = 0$.
%   \item[(b)] If $Y$ is a plane curve of degree $d$, show that $p_a(Y) = \frac{1}{2}(d - 1)(d - 2)$.
%   \item[(c)] More generally, if $H$ is a hypersurface of degree $d$ in $\PP^n$, then $p_a(H) = {d - 1 \choose n}$.
%   \item[(d)] If $Y$ is a complete intersection of surfaces of degrees $a$, $b$ in $\PP^3$, then $p_a(Y) = \frac{1}{2}ab(a + b - 4) + 1$.
%   \item[(e)] Let $Y^r \subseteq \PP^n$, $Z^s \subseteq \PP^m$ be projective varieties, and embed $Y \times Z \subseteq \PP^n \times \PP^m \to \PP^N$ by the Segre embedding. Show that
%   \begin{equation*}
%     p_a(Y \times Z) = p_a(Y)p_a(Z) + (-1)^s p_a(Y) + (-1)^r p_a(Z).
%   \end{equation*}
% \end{itemize}
