\documentclass[12pt]{article}
\usepackage[margin=1in]{geometry}
\usepackage[T1]{fontenc}
\usepackage[tracking]{microtype}
\usepackage[sc,osf]{mathpazo}
\linespread{1.025}
\usepackage[euler-digits,small]{eulervm}
\AtBeginDocument{\renewcommand{\hbar}{\hslash}}
\usepackage{amsmath}
\usepackage{amsthm}
\usepackage{amssymb}
\usepackage{amsbsy}
\usepackage{bbm}
\usepackage{setspace}
\usepackage{enumitem}
\usepackage{graphicx}
\usepackage{float}
\usepackage{multicol}
\usepackage{tikz-cd}
\usepackage{yfonts}
\usepackage{mathrsfs}
\usepackage{pgfplots}
\usepackage{mathtools}
\DeclarePairedDelimiterX{\infdivx}[2]{(}{)}{%
  #1\;\delimsize\|\;#2%
}
\newcommand{\infdiv}{D\infdivx}
\DeclarePairedDelimiter{\norm}{\lVert}{\rVert}
\newcommand{\prob}[1]{\noindent {\bf #1}.}
\newcommand{\abs}[1]{\left| #1  \right|}
\newcommand{\goth}[1]{\textfrak{#1}}
\newcommand{\spec}[1]{\text{Spec}~#1}
\newcommand{\Ker}[1]{\text{ker}~#1}
\newcommand{\Img}[1]{\text{im}~#1}
\newcommand{\Coker}[1]{\text{Coker}(#1)}
\newcommand{\Supp}[1]{\text{Supp}(#1)}
\newcommand{\height}[1]{\text{height}~#1}
\newcommand{\A}{\mathbb{A}}
\newcommand{\OO}{\mathcal{O}}
\newcommand{\R}{\mathbb{R}}
\newcommand{\G}{\mathbb{G}}
\newcommand{\C}{\mathbb{C}}
\newcommand{\Q}{\mathbb{Q}}
\newcommand{\N}{\mathbb{N}}
\newcommand{\Z}{\mathbb{Z}}
\newcommand{\PP}{\mathbb{P}}
\newcommand{\pd}[2]{\frac{\partial #1}{\partial #2}}
\DeclareMathOperator{\Hom}{Hom}
\newtheorem{prop}{Proposition}
\newtheorem{definition}{Definition}
\newtheorem{theorem}{Theorem}
\theoremstyle{definition}
\newtheorem{exercise}{Exercise}
\newtheorem{example}{Example}
\newtheorem{corollary}{Corollary}
\newtheorem{lemma}{Lemma}

\begin{document}
\begin{enumerate} [label=\textbf{\arabic*.}, leftmargin=-0.05em]

\item Recall that a curve is \textit{rational} if it is birationally equivalent to $\PP^1$. Let $Y$ be a nonsingular rational curve which is not isomorphic to $\PP^1$.
\begin{itemize}
    \item[(a)] Show that $Y$ is isomorphic to an open subset of $\A^1$.
    \item[(b)] Show that $Y$ is affine.
    \item[(c)] Show that $A(Y)$ is a unique factorization domain.
\end{itemize}

\begin{proof} $ $ \vspace{0pt}
  \begin{itemize}
    \item[(a)]
    \item[(b)]
    \item[(c)]
  \end{itemize}
\end{proof}

\item \textit{An Elliptic Curve.} Let $Y$ be the curve $y^2 = x^3 - x$ in $\A^2$, and assume that the characteristic of the base field $k$ is $\neq 2$. In this exercise we will show that $Y$ is not a rational curve, and hence $K(Y)$ is not a pure transcendental extension of $k$.
\begin{itemize}
    \item[(a)] Show that $Y$ is nonsingular, and deduce that $A = A(Y) \simeq k[x, y] / (y^2 - x^3 + x)$ is an integrally closed domain.
    \item[(b)] Let $k[x]$ be the subring of $K = K(Y)$ generated by the image of $x$ in $A$. Show that $k[x]$ is a polynomial ring, and that $A$ is the integral closure of $k[x]$ in $K$.
    \item[(c)] Show that there is an automorphism $\sigma : A \to A$ which sends $y$ to $-y$ and leaves $x$ fixed. For any $a \in A$, define the \textit{norm} of $a$ to be $N(a) = a \sigma(a)$. Show that $N(a) \in k[x]$, $N(1) = 1$, and $N(ab) = N(a) \cdot N(b)$ for any $a, b \in A$.
    \item[(d)] Using the norm, show that the units in $A$ are precisely the nonzero elements of $k$. Show that $x$ and $y$ are irreducible elements of $A$. Show that $A$ is \textit{not} a unique factorization domain.
    \item[(e)] Prove that $Y$ is not a rational curve. 
\end{itemize}

\begin{proof} $ $ \vspace{0pt}
  \begin{itemize}
    \item[(a)] Let $f(x, y) = y^2 - x^3 + x$. Then, we have
    \begin{equation*}
      \pd{f}{x} = -3x^2 + 1, \quad \pd{f}{y} = 2y.
    \end{equation*}
    If $(\partial f / \partial y)(P) = 0$ for some $P = (a, b) \in Y$, then $b = 0$, so either $a = 0$ or $a = \pm 1$. In any case, $(\partial f / \partial x)(P) \neq 0$ since $\text{char}~k \neq 2$. Therefore, both partial derivatives of $f$ do not vanish for any $P \in Y$, hence $Y$ is nonsingular. This means the local ring $A_\goth{m}$ for every maximal ideal $\goth{m}$ of $A$ is a regular local ring, and since $A$ has dimension one by Krull's Hauptidealsatz, $A_\goth{m}$ also has dimension one, hence $A_\goth{m}$ is a integrally closed by (6.2A). Since being integrally closed is a local property by [AM p.63], $A$ is also integrally closed.

    \item[(b)] Since $k$ is algebraically closed and $x \notin k$, $x$ is transcendental over $k$, hence $k[x]$ is a polynomial ring. Since $y^2 - x^3 + x = 0$ in $K$, $y$ is the integral closure of $k[x]$ in $K$. Thus, the integral closure of $k[x]$ contains $A$, and $A$ itself is integrally closed, hence $A$ is the integral closure of $k[x]$.

    \item[(c)] The map $\sigma$ is clearly bijective. To show it is an automorphism of $A$, it suffices to show $\sigma$ as a map from $k[x, y]$ to itself fixes $y^2 - x^3 + x$, which it indeed does. An element of $A$ is of the form $a = f + yg$ for some $f, g \in k[x]$, hence we have
    \begin{equation*}
      N(a) = a \sigma(a) = (f + yg)(f - yg) = f^2 - y^2g^2 = f^2 - (x^3 - x)g^2 \in k[x].
    \end{equation*}
    The map $\sigma$ is an isomorphism, so it fixes $k$; in particular it fixes $1$, hence $N(1) = 1 \cdot 1 = 1$. Lastly, if $a, b \in A$, then
    \begin{equation*}
      N(ab) = (ab) \sigma(ab) = (ab)(\sigma(a) \sigma(b)) = (a \sigma(a))(b \sigma(b)) = N(a) \cdot N(b).
    \end{equation*}

    \item[(d)] Let $u$ be a unit in $A$ and let $u^{-1}$ be its inverses. By (c), we have $N(u) N(u^{-1}) = 1$, so $N(u)$, $N(u^{-1})$ are units in $k[x]$. Since $k[x]$ is a polynomial ring, the units are precisely the nonzero elements of $k$, that is $N(u) \in k$. Since the norm fixes the degree, we must have $u \in k$.

    Suppose $x = ab$ for some $a, b \in A$. Then, $x^2 = N(a) N(b)$ and $x$ is irreducible in $k[x]$, so either $N(a)$, $N(b)$ each are associates with $x$, or $N(a)$ is an associate of $x^2$ and $N(b)$ is a unit. It suffices to show $N$ is not onto $k[x]$, that is there does not exist $c \in A$ such that $N(c) = x$. By the formula above, the degree of $f^2$ is even while $(x^3 - x)g^2$ has degree $0$ or an odd number, so $f^2 - (x^3 - x)g^2$ must have degree greater than $1$. Therefore, $N(b)$ is a unit, hence $b$ is a unit. The case for $y$ follows similarly, since $N(y) = x(1 + x)(1 - x)$, so if $y = ab$, then either $N(a)$ is associates with $x$ or $x(1 + x)(1 - x)$, and the former case was shown to be impossible.
    
    $A$ is not a unique factorization domain since $y^2 = x(x + 1)(x - 1)$, and $x$, $y$ are irreducible and hence prime but are not associates since the only units in $A$ are the nonzero elements of $k$.
  \end{itemize}
\end{proof}

\item Show by example that the result of (6.8) is false if either (a) $\dim{X} \geq 2$, or (b) $Y$ is not projective.

\begin{proof} $ $ \vspace{0pt}
  \begin{itemize}
    \item[(a)] Consider the morphism $\A^2 - O \to \PP^1$ defined by mapping $(x, y)$ to a point in $\PP^1$ with homogenous coordinates $(x, y)$, where $O$ is the origin.

    \item[(b)] Let $X$ be an abstract nonsingular curve isomorphic to $\PP^1$ and write $\PP^1 = \A^1 \cup \{ \infty \}$. Then, we have an isomorphism $\varphi : \PP^1 - \infty \to \A^1$; however, $\varphi$ clearly cannot be extended to $\infty$.
  \end{itemize}
\end{proof}
\filbreak

\item Let $Y$ be a nonsingular projective curve. Show that every nonconstant rational function $f$ on $Y$ defines a surjective morphism $\varphi : Y \to \PP^1$, and that for every $P \in \PP^1$, $\varphi^{-1}(P)$ is a finite set of points.

\begin{proof}
  Identifying $k$ with the affine line, $f$ is a rational map between quasi-projective curves. We first show that if $f$ is nonconstant, then, $f$ is a dominant rational map into $\A^1$. Since the closed subsets of $\A^1$ are either finite subsets or the entire line $\A^1$, it suffices to show $\Img{f}$ is an infinite set of points. If $\Img{f}$ is a finite set of points, then it must be of cardinality greater than $1$ since $f$ is nonconstant. Every nonsingular projective curve is isomorphic to an abstract nonsingular curve, so the closed subsets of $Y$ are either finite set of points or the entire curve. If $\Img{f} = \{ P_1, \dots, P_n \}$, $n > 1$, then $f^{-1}(P_i)$ cannot be the entire curve $Y$ for any $1 \leq i \leq n$, thus $Y = f^{-1}(\Img{f}) = f^{-1}(P_1) \cup \dots \cup f^{-1}(P_n)$, which is a finite union of finite sets, implying $Y$ is a finite set of points. This is clearly not true by Exercise 4.8. Therefore, $\Img{f}$ is infinite set of points, so its closure is the entire affine line, hence $f$ is a dominant rational map.

  Then, $\A^1$ is a rational curve, in particular it is birationally equivalent to $\PP^1$, so if $U$ is the largest open affine subset of $Y$ such that $f : U \to \A^1$ is defined as a morphism by Exercise 4.2, we have a dominant morphism $f' : U \to \A^1 \hookrightarrow \PP^1$. Since $U$ is open and nonempty, its complement $Y - U$ is closed and proper subset of $Y$, hence it is a finite set of points, so by (6.8) we can extend $f'$ to be a dominant morphism $\varphi: Y \to \PP^1$ between nonsingular projective curves. 
  
  It remains to show $\varphi$ is surjective. A dominant morphism $\varphi : Y \to \PP^1$ induces a $k$-homomorphism $\varphi^* : K(\PP^1) \to K(Y)$ between function fields, where $K(\PP^1) \simeq k(t)$ for some indeterminant $t$. Since $k(t)$ and $K(Y)$ both have transcendence degree $1$, $\varphi$ is injective; in particular every valuation ring of $k(t)$ can be extended to one of $K(Y)$ since $K(Y)$ is integrally closed over $k(t)$. The map $\varphi$ induces a morphism between abstract nonsingular curves $\varphi_\# : C_{K(Y)} \to C_{k(x)}$, where for $P \in C_{K(Y)}$ we have $\varphi_\# (P) = P \cap C_{k(x)}$ (the point $P$ can be identified with a valuation ring in $K(Y)$). Every valuation ring of $k(t)$ is the intersection of some valuation ring of $K(Y)$ and $k(t)$, which implies $\varphi_\#$ is surjective. Since we have the following commutative diagram
  \[ \begin{tikzcd}
    Y \arrow[d] \arrow[r, "\varphi"]  & \PP^1 \arrow[d] \\
    C_{K(Y)} \arrow[r, "\varphi_\#"'] & C_{k(t)}       
    \end{tikzcd} \]
  where the vertical arrows are isomorphisms, $\varphi$ is also surjective.

  Every nonsingular projective curve is isomorphic to an abstract nonsingular curve, where its closed sets are either finite set of points or the entire curve. This implies $\varphi^{-1}(P)$ is either finite set of points or all of $Y$, and it cannot be $Y$ since $\varphi$ is surjective onto $\PP^1$.
\end{proof}

\end{enumerate}
\end{document}
