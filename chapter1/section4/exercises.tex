\documentclass[12pt]{article}
\usepackage[margin=1in]{geometry}
\usepackage[T1]{fontenc}
\usepackage[tracking]{microtype}
\usepackage[sc,osf]{mathpazo}
\linespread{1.025}
\usepackage[euler-digits,small]{eulervm}
\AtBeginDocument{\renewcommand{\hbar}{\hslash}}
\usepackage{amsmath}
\usepackage{amsthm}
\usepackage{amssymb}
\usepackage{amsbsy}
\usepackage{bbm}
\usepackage{setspace}
\usepackage{enumitem}
\usepackage{graphicx}
\usepackage{float}
\usepackage{multicol}
\usepackage{tikz-cd}
\usepackage{yfonts}
\usepackage{mathrsfs}
\usepackage{pgfplots}
\newcommand{\prob}[1]{\noindent {\bf #1}.}
\newcommand{\abs}[1]{\left| #1  \right|}
\newcommand{\goth}[1]{\textfrak{#1}}
\newcommand{\spec}[1]{\text{Spec}~#1}
\newcommand{\Ker}[1]{\text{Ker}~#1}
\newcommand{\Img}[1]{\text{Im}~#1}
\newcommand{\Coker}[1]{\text{Coker}(#1)}
\newcommand{\Supp}[1]{\text{Supp}(#1)}
\newcommand{\height}[1]{\text{height}~#1}
\newcommand{\A}{\mathbb{A}}
\newcommand{\OO}{\mathcal{O}}
\newcommand{\R}{\mathbb{R}}
\newcommand{\G}{\mathbb{G}}
\newcommand{\C}{\mathbb{C}}
\newcommand{\Q}{\mathbb{Q}}
\newcommand{\N}{\mathbb{N}}
\newcommand{\Z}{\mathbb{Z}}
\newcommand{\PP}{\mathbb{P}}
\DeclareMathOperator{\Hom}{Hom}
\newtheorem{prop}{Proposition}
\newtheorem{definition}{Definition}
\newtheorem{theorem}{Theorem}
\theoremstyle{definition}
\newtheorem{exercise}{Exercise}
\newtheorem{example}{Example}
\newtheorem{corollary}{Corollary}
\newtheorem{lemma}{Lemma}

\begin{document}
\begin{enumerate} [label=\textbf{\arabic*.}, leftmargin=-0.05em]

\item If $f$ and $g$ are regular functions on open subsets $U$ and $V$ of a variety $X$, and if $f = g$ on $U \cap V$, show that the function which is $f$ on $U$ and $g$ on $V$ is a regular function on $U \cup V$. Conclude that if $f$ is a \textit{rational} function on $X$, then there is a largest open subset $U$ of $X$ on which $f$ is represented by a regular function. We say that $f$ is \textit{defined} at the points of $U$.

\begin{proof}
    Denote the function which is $f$ on $U$ and $g$ on $V$ as $F$. It suffices to show for all $P \in U \cap V$ that there exists an open neighborhood $W$ of $P$ contained in $U \cap V$ such that $F$ is a rational function on $W$. Since $U \cap V$ is an open subset of $U$ and $V$, for all $P \in U \cap V$ there exists open subsets $A$ of $U$ and $B$ of $W$ such that $F$ is rational with $h(Q), h(Q) \neq 0$ for all $Q \in A \cap B$ and $F = f = g$ on $A \cap B$, hence $W = A \cap B \subseteq U \cap V$ is the desired open subset.
    
    If $f$ is a rational function on $X$, then the set of open subsets of $X$ on which $f$ is represented by regular function is nonempty, and since $X$ is a noetherian space, there is a maximal element in this set. Then, the largest open subset $U$ of $X$ on which $f$ is represented by a regular function is the union of maximal elements in this subset of open subsets.
\end{proof}

\item Same problem for rational maps. If $\varphi$ is a rational map of $X$ to $Y$, show there is a largest open set on which $\varphi$ is represented by a morphism. We say the rational map is \textit{defined} at the points of that open set.

\begin{proof}
    Let $\varphi$ and $\psi$ be morphisms on open subsets $U$ and $V$ of a variety $X$ to a variety $Y$, and suppose $\varphi = \psi$ on $U \cap V$. If $f$ is a regular function on $Y$, then $\varphi^*f$ and $\psi^*f$ are regular functions on the open subsets $U$ and $V$ which agree on $U \cap V$, so by Exercise 1 the function which is $\varphi^*f$ on $U$ and $\psi^*f$ on $V$ is a regular function $U \cap V$, i.e. the map that is $\varphi$ on $U$ and $\psi$ on $V$ is indeed a rational map.

    If $\varphi : X \to Y$ is a rational map, then the set of open subsets of $X$ on which $\varphi$ is represented by morphism function is nonempty, and since $X$ is a noetherian space, there is a maximal element in this set. Then, the largest open subset $U$ of $X$ on which $\varphi$ is represented by a morphism is the union of maximal elements in this subset of open subsets.
\end{proof}

\item A variety $Y$ is \textit{rational} if it is birationally equivalent to $\PP^n$ for some $n$ (or, equivalently by (4.5), if $K(Y)$ is a pure transcendental extension of $k$).
\begin{itemize}
    \item[(a)] Any conic in $\PP^2$ is a rational curve.
    \item[(b)] The cuspidal cubic $y^2 = x^3$ is a rational curve.
    \item[(c)] Let $Y$ be the nodal cubic curve $y^2 z = x^2(x + z)$ in $\PP^2$. Show that the projection $\varphi$ from the point $P = (0, 0, 1)$ to the line $z = 0$ induces a birational map from $Y$ to $\PP^1$. Thus, $Y$ is a rational curve.
\end{itemize}

\begin{proof} $ $ \vspace{0pt}
    \begin{itemize}
        \item[(a)] A conic in $\PP^2$ can be covered by open affine varieties that are either isomorphic to $y = x^2$ or $xy = 1$. The former is isomorphic to $\A^1$, hence it is isomorphic to an open subset of $\PP^1$, hence it is birationally equivalent to $\PP^1$. The latter has function field isomorphic to $k(x)$, hence it is birationally equivalent to $\A^1$, hence it is also birationally equivalent to $\PP^1$.
        \filbreak

        \item[(b)] The cuspidal cubic has coordinate ring $k[t^2, t^3]$, so its function field is $k(t)$, hence it is birationally equivalent to $\A^1$, hence it is birationally equivalent to $\PP^1$.

        \item[(c)] The line $z = 0$ in $\PP^2$ corresponds to a hyperplane isomorphic to $\PP^1$, so the projection $\varphi : \PP^2 - \{ P \} \to \PP^1$ is a morphism. In coordinates, $\varphi$ is defined as $(x_0, x_1, x_2) \mapsto (x_0, x_1)$, so $\varphi$ induces a morphism from $Y - P$ to $\PP^1$. Thus, $\varphi(Y - P)$ is the set of all lines in $\A^2$ that pass through the origin and a point in the affine nodal curve $y^2 = x^3 + x^2$. This is an open set in $\PP^1$ isomorphic to $\A^1$ since it contains all lines in $\A^2$ besides the one defined by $x = \pm y$. To further elaborate, if $(x, y) \in \PP^1$ with $x \neq \pm y$ and $x \neq 0$, say, then write $y = \lambda x$ with $\lambda \neq \pm 1$, so we have
        \begin{equation*}
            \lambda^2 x^2 = x^3 + x^2 \implies x = \lambda^2 - 1,
        \end{equation*}
        that is there exists a line in $\A^2$ passing through a point in $y^2 + x^3 + x^2$ with slope $y / x$. Rephrasing, we have shown that the map $\varphi : Y - P \to \PP^1$ is surjective besides at the two points $(1, 1)$ and $(1, -1)$. It is also injective since the $x_2$-coordinate can be completely determined by the values of $(x_0, x_1)$, that is if $x_0 \neq 0$, then setting $x_0 = 1$, we have
        \begin{equation*}
            x_1^2 x_2 = 1 + x_2 \implies x_2 = \frac{1}{x_1^2 - 1}\text{ or } x_2 = 0,
        \end{equation*}
        and $x_1 \neq \pm 1$ since $x_0 \neq \pm x_1$, and $x_2 \neq 0$ since the only point on $Y$ with $x_2 = 0$ is $P$, and $\varphi$ is restricted $Y - P$. Hence, $\varphi$ is an isomorphism of the open subset $Y - P$ of $Y$ to the open subset $\PP^1 - \{ (1, 1), (1, -1) \}$ in $\PP^1$, hence $Y$ is birationally equivalent to $\PP^1$.
    \end{itemize}
\end{proof}

\item Show that the quadric surface $Q : xy = zw$ in $\PP^3$ is birational to $\PP^2$, but not isomorphic to $\PP^2$.

\begin{proof}
   $Q$ is the Segre embedding of $\PP^1 \times \PP^1$ in $\PP^3$, and $\PP^1 \times \PP^1$ contains a copy of $\A^1 \times \A^1$, so $Q$ contains a copy of $\A^2$, hence $Q$ is birational to $\PP^2$. It is an axiom of projective geometry that any two lines intersect in $\PP^2$; however, it was shown in Exercise 2.15 that there exists lines in $Q$ that do not intersect, hence $Q$ and $\PP^2$ cannot be isomorphic.
\end{proof}

\item Let $Y$ be the cuspidal cubic curve $y^2 = x^3$ in $\A^2$. Blow up the point $O = (0, 0)$, let $E$ be the exceptional curve, and let $\tilde{Y}$ be the strict transform of $Y$. Show that $E$ meets $\tilde{Y}$ in one point, and that $\tilde{Y} \cong \A^1$. In this case the morphism $\varphi : \tilde{Y} \to Y$ is bijective and bicontinuous, but it is not an isomorphism.

\begin{proof}
    Let $t$, $u$ be homogenous coordinates for $\PP^1$. Then $X$, the blowing-up of $\A^2$ at $O$, is defined by the equation $xu = ty$ inside $\A^2 \times \PP^1$. We obtain the total inverse image of $Y$ in $X$ by considering the equations $y^2 = x^3$ and $xu = ty$ in $\A^2 \times \PP^1$. Now $\PP^1$ is covered by the open sets $t \neq 0$ and $u \neq 0$, which we consider separately. If $t \neq 0$, we can set $t = 1$, and use $u$ as an affine parameter. Then we have the equations
    \begin{equation*}
        y^2 = x^3, \quad y = xu
    \end{equation*}
    in $\A^3$ with coordinates $x$, $y$, $u$. Substituting, we get $x^2u^2 - x^3 = 0$, which factors. Thus, we obtain two irreducible components, one defined by $x = 0$, $y = 0$, $u$ arbitrary, which is $E$, and the other defined by $u = x$ and $y = ux$. This is $\tilde{Y}$, and $\tilde{Y}$ meets $E$ at the point $u = 0$. Similarly, if $u \neq 0$, then we can set $u = 1$, and use $t$ as an affine parameter to obtain the equations
    \begin{equation*}
        y^2 = x^3, \quad x = ty
    \end{equation*}
    in $\A^3$ with coordinates $x$, $y$, $t$. Substituting, we get $y^2 = t^3 y^3$, which factors as well. Besides the exceptional curve, we have the component defined by $1 - t^3 y = 0$ and $x = ty$, which does not meet $E$. Hence, $E$ meets $\tilde{Y}$ at only $(1, 0) \in E$. This also show $\tilde{Y}$ is contained in the open set defined by $t \neq 0$, so it is isomorphic to the affine variety in $\A^3$ defined by $u = x$ and $y = ux$, which isomorphic to $y = x^2$ in $\A^2$, hence $\tilde{Y} \simeq \A^1$.
\end{proof}

\end{enumerate}

\end{document}
