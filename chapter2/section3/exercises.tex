\documentclass{article}
\usepackage[margin=0.75in]{geometry}
\usepackage{amsmath}
\usepackage{amsthm}
\usepackage{amssymb}
\usepackage{enumitem}
\usepackage{tikz-cd}
\usepackage{yfonts}
\usepackage{mathrsfs}
\usepackage{xcolor}
\usepackage{physics}

\DeclareMathAlphabet{\mathpzc}{OT1}{pzc}{m}{it}

\newcommand{\goth}[1]{\mathfrak{#1}}
\newcommand{\reduced}[1]{#1_{\text{red}}}
\newcommand{\fF}{\mathscr{F}}
\newcommand{\fG}{\mathscr{G}}
\newcommand{\fE}{\mathscr{E}}
\newcommand{\fO}{\mathscr{O}}
\newcommand{\fL}{\mathscr{L}}
\newcommand{\fM}{\mathscr{M}}
\newcommand{\fI}{\mathscr{I}}
\newcommand{\fT}{\mathscr{T}}
\newcommand{\fK}{\mathscr{K}}
\newcommand{\fS}{\mathscr{S}}
\newcommand{\fN}{\mathscr{N}}
\newcommand{\fJ}{\mathscr{J}}
\newcommand{\fR}{\mathscr{R}}
\newcommand{\fH}{\mathscr{H}}
\newcommand{\PP}{\mathbb{P}}
\newcommand{\gm}{\goth{m}}
\newcommand{\A}{\mathbb{A}}
\newcommand{\R}{\mathbb{R}}
\newcommand{\C}{\mathbb{C}}
\newcommand{\Q}{\mathbb{Q}}
\newcommand{\N}{\mathbb{N}}
\newcommand{\Z}{\mathbb{Z}}
\newcommand{\G}{\mathbb{G}}
\newcommand{\gF}{\goth{F}}
\newcommand\srestr[2]{{\left.\kern-\nulldelimiterspace #1\vphantom{\small|} \right|_{#2}}}
\newcommand\restr[2]{{\left.\kern-\nulldelimiterspace #1 \vphantom{\big|} \right|_{#2}}}

\newtheorem{theorem}{Theorem}
\newtheorem{lemma}{Lemma}
\newtheorem{corollary}{Corollary}

\DeclareMathOperator{\id}{id}
\DeclareMathOperator{\bProj}{\mathpzc{Proj}}
\DeclareMathOperator{\Frac}{Frac}
\DeclareMathOperator{\rk}{rank}
\DeclareMathOperator{\pic}{Pic}
\DeclareMathOperator{\cacl}{CaCl}
\DeclareMathOperator{\trd}{tr.d.}
\DeclareMathOperator{\cl}{Cl}
\DeclareMathOperator{\depth}{depth}
\DeclareMathOperator{\codim}{codim}
\DeclareMathOperator{\Div}{Div}
\DeclareMathOperator{\coker}{coker}
\DeclareMathOperator{\len}{length}
\DeclareMathOperator{\height}{height}
\DeclareMathOperator{\supp}{Supp}
\DeclareMathOperator{\proj}{Proj}
\DeclareMathOperator{\im}{im}
\DeclareMathOperator{\Hom}{Hom}
\DeclareMathOperator{\Der}{Der}
\DeclareMathOperator{\spec}{Spec}

\author{James Lee}
\pagecolor[RGB]{20,20,20}
\color[RGB]{255,255,255}

\title{Chapter 2, Section 3}

\begin{document}
\maketitle
\begin{enumerate} [label=\textbf{\arabic*.}, leftmargin=0em]

\item[\textbf{1.}] Show that a morphism $f : X \to Y$ is locally of finite type if and only if for every open affine subset $V = \spec{B}$ of $Y$, $f^{-1}(V)$ can be covered by open affine subsets $U_j = \spec{A_j}$, where each $A_j$ is a finitely generated $B$-algebra.

\begin{proof}
    We reduce to proving the following statement: let $f : X \to Y$ be a morphism of schemes with $Y = \spec{B}$, which can be covered by open affine subsets $V_i = \spec{B_i}$, such that for each $i$, $f^{-1}(V_i)$ can be covered by open affine subsets $U_{ij} = \spec{A_{ij}}$, where each $A_{ij}$ is a finitely generated $B_i$-algebra. Then $X$ can be covered by open affine subsets $U_k = \spec{A_k}$, where each $A_k$ is a finitely generated $B$-algebra.
    
    We show each $B_i$ can be chosen to be a finitely generated $B$-algebra, then by restriction of scalars each $A_{ij}$ is a finitely generated $B$-algebra, which proves the statement. For each $i$, there exists $g_i \in B$ such that $D(g_i) \subseteq V_i$, where $D(g_i) \cong \spec{B_{g_i}}$. Then $f^{-1}(D(g_i)) \subseteq f^{-1}(V_i)$, so we can cover $f^{-1}(D(g_i))$ by open affines of the form $\text{Spec}((A_{ij})_{h_k})$ with $h_k \in A_{ij}$. Since each $(A_{ij})_{h_k}$ is a finitely generated $A_{ij}$-algebra, and each $(A_{ij})_{h_k}$ is a finitely generated $B_{g_i} \cong (B_i)_{\overline{g}_i}$, where each $\overline{g}_i$ is the image of $g_i$ in $B_i$, what we have shown is each $f^{-1}(D(g_i))$ can be covered by open affine subsets $U_{ij}' = \spec{A'_{ij}}$, where each $A_{ij}'$ is a finitely generated $B_{g_i}$-algebra. Each $B_{g_i}$ is a finitely generated $B$-algebra, generated by $1$ and $1 / g_i$, which is what we wanted to show.
\end{proof}

\item[\textbf{2.}] A morphism $f : X \to Y$ of schemes is \textit{quasi-compact} if there is a cover of $Y$ by open affines $V_i$ such that $f^{-1}(V_i)$ is quasi-compact for each $i$. Show that $f$ is quasi-compact if and only if for \textit{every} open affine subset $V \subseteq Y$, $f^{-1}(V)$ is quasi-compact.

\begin{proof}
    We reduce to proving the following statement: let $f : X \to Y$ be a morphism of schemes with $Y$ affine, which can be covered by open subsets $V_i$ such that $f^{-1}(V_i)$ is quasi-compact in $X$. Then $X$ is quasi-compact. Indeed, $Y$ is quasi-compact, so a finite number of $i$ will do, and a finite union of quasi-compact sets is quasi-compact, and $X = f^{-1}(Y) = f^{-1}(\bigcup_{i = 1}^n V_i) = \bigcup_{i = 1}^n f^{-1}(V_i)$, hence $X$ is quasi-compact.
\end{proof}

\item[\textbf{3.}]
\begin{itemize}
    \item[(a)] Show that a morphism $f : X \to Y$ is of finite type if and only if it is locally of finite type and quasi-compact.
    \item[(b)] Conclude from this that $f$ is of finite type if and only if for \textit{every} open affine subset $V = \spec{B}$ of $Y$, $f^{-1}(V)$ can be covered by a finite number of open affines $U_j = \spec{A_j}$, where each $A_j$ is a finitely generated $B$-algebra.
    \item[(c)] Show also if $f$ is of finite type, then for \textit{every} open affine subset $V = \spec{B} \subseteq Y$, and for \textit{every} open affine subset $U = \spec{A} \subseteq f^{-1}(V)$, $A$ is a finitely generated $B$-algebra.
\end{itemize}

\begin{proof} $ $ \vspace{0pt}
\begin{itemize} [leftmargin=0cm]
    \item[(a)] Suppose $f : X \to Y$ is of finite type. Then by definition it is also locally of finite type, so there exists a covering of $Y$ by open affine subsets $V_i = \spec{B_i}$, such that each $i$, $f^{-1}(V_i)$ can be covered by a finite number of open affine subsets $U_{ij} = \spec{A_{ij}}$, where each $A_{ij}$ is a finitely generated $B_i$-algebra. Each $f^{-1}(V_i)$ is quasi-compact since every affine scheme is quasi-compact and $f^{-1}(V_i)$ is a finite union of quasi-compact sets. Conversely, if $f : X \to Y$ is locally of finite type and quasi-compact, then each $f^{-1}(V_i)$ is quasi-compact, so it can be covered by a finite number of open affines $U_{ij} = \spec{A_{ij}}$, where each $A_{ij}$ is a finitely generated $B_i$-algebra.

    \item[(b)] A morphism $f : X \to Y$ is of finite type if and only if it is locally of finite type and quasi-compact if and only if for every open affine subset $V = \spec{B}$ of $Y$, $f^{-1}(V)$ is quasi-compact and can be covered by open affines $U_j = \spec{A_j}$, where each $A_j$ is a finitely generated $B$-algebra if and only if for every open affine subset $V = \spec{B}$ of $Y$, $f^{-1}(V)$ can be covered by a finite number of open affines $U_j = \spec{A_j}$, where each $A_j$ is a finitely generated $B$-algebra.

    \item[(c)] By restricting the domain of $f$, it suffices to prove the following: if $X$ is a scheme over a ring $B$ that can be covered by a finite number of open subsets that are the spectra of finitely generated $B$-algebras, then for every open affine $U = \spec{A} \subseteq X$, $A$ is a finitely generated $B$-algebra. By Nike's trick, we can cover $U$ by the distinguished basic sets $\spec{A_f}$, where each $A_f$ is a finitely generated $B$-algebra. Since $U$ is quasi-compact, a finite number will do.
    
    So now, we have reduced to the following purely algebraic problem: $A$ is a $B$-algebra, $f_1, \dots, f_r$ are a finite number of elements of $A$, which generate the unit ideal, and each localization $A_{f_i}$ is a finitely generated $B$-algebra. Then $A$ is finitely generated over $B$. Write $\sum_{i = 1}^r g_i f_i$. Let $S$ be the union of $\{ g_1, \dots, g_r, f_1, \dots, f_r \}$ and a finite subset of $A$ such that the image of it under the natural map $A \to A_{f_i}$ generates $A_{f_i}$ as a $B$-algebra for all $i$, and let $B[S]$ be the $B$-subalgebra of $A$ generated by $S$. Localization is left-exact, so $B[S]_{f_i} \to A_{f_i}$ is injective for all $i$, and it is surjective as well by construction of $S$. Thus, the inclusion $B[S] \to A$ viewed as a $B[S]$-module homomorphism is an isomorphism by the following statement from commutative algebra: let $R$ be any ring, let $M$ by any $R$-module, and let $h_1, \dots, h_n$ be elements of $R$ that generate the unit ideal. If $M_{h_i} = 0$ for all $i$, then $M = 0$. Indeed, if $\goth{p}$ is any prime ideal of $R$, then $h_i \notin \goth{p}$ for some $i$. Denote $\bar{h}_i$ the image of $h_i$ in $A_\goth{p}$. Then, $M_{\goth{p}} = (M_\goth{p})_{\bar{h}_i} = (M_{h_i})_{\goth{p} R_{h_i}} = 0$ (A.M. \S 3).
\end{itemize}
\end{proof}

\begin{theorem} [Nike's Trick] Let $X$ be a scheme, and let $U_i = \spec{A_i}$, $i = 1, 2$, be open affine subsets of $X$. Then there is an open cover of $U_1 \cap U_2$ consisting of open sets that are distinguished basic open sets in both $U_i$.

\begin{lemma}
    If $(f, f^\#) : (X, \fO_X) \to (Y, \fO_Y)$ is a morphism of locally ringed spaces, then $f^{-1}(Y_g) = X_{f^\#(g)}$ for any $g \in \Gamma(Y, \fO_Y)$.
\end{lemma}

\begin{proof}
    Pick $g \in A_1$ such that $\spec{(A_1)_g} = (U_1)_g \subseteq U_1 \cap U_2$, and set $U_1 = (U_1)_g$, so we have an open immersion $\iota : U_1 \hookrightarrow U_2$, which induces a ring homomorphism $\varphi : A_2 \to (A_1)_g$. Pick $h \in A_2$ such that $(U_2)_h \subseteq U_1$. By lemma 1, $\iota^{-1}((U_2)_h) = (U_1)_{\varphi(h)}$, and since $\iota$ is an open immersion, $(U_1)_{\varphi(h)} \cong (U_2)_h$ (we can assume $\varphi(h) \in A$).
\end{proof}

\end{theorem}

\item[\textbf{4.}] Show that a morphism $f : X \to Y$ is finite if and only if for \textit{every} open affine subset $V = \spec{B}$ of $Y$, $f^{-1}(V)$ is affine, equal to $\spec{A}$, where $A$ is a finite $B$-module.

\begin{proof}
    Let $f : X \to Y$ be a finite morphism of schemes. Then there exists an open covering of $Y$ by sets $V_i = \spec{B_i}$, such that for each $i$, $f^{-1}(V_i)$ is affine, equal to $\spec{A_i}$, where $A_i$ is a finite $B_i$-module. Let $V = \spec{B}$ be an open subset of $Y$. By Nike's trick, we can cover $V$ by open affines that are distinguished open set in $V$ and some $V_i$, i.e., open sets of the form $\spec{B_g} = \spec{(B_i)_{h}}$ for some $g \in B, h \in B_i$. Since $V$ is quasi-compact, a finite number will do. By lemma 1, $f^{-1}(\spec(B_i)_h) = \spec{(A_i)_h} \subseteq f^{-1}(V)$. Since $(A_i)_h$ is a finite $(B_i)_h$-module and $B_g \cong (B_i)_h$, we deduce that there exists a finite cover of $V$ by basic open sets $\spec{B_{g_j}}$ for some $g_j \in B$ such that $f^{-1}(\spec{B_{g_j}}) = \spec{C_j}$, where each $C_j$ is a finite $B_{g_j}$-module. We show $X' = f^{-1}(V)$ is affine using the criterion in (Ex. 2.17b). The restriction of $f$ to $X' \to \spec{B}$ induces a ring homomorphism $B \to A = \Gamma(X', \fO_X)$. Denote $\bar{g}_i$ the image of $g_i$ in $A$. Since $g_i$ generate the unit ideal in $B$, its image in $A$ also generate the unit ideal. Also, by lemma 1 $X'_{\bar{g}_j} = f^{-1}(\spec{B_{g_j}}) = \spec{C_j}$, hence $X' = \spec{A}$. It remains to show $A$ is a finite $B$-module, which we reduce to the following algebraic problem: let $A$ be a $B$-algebra, let $g_i \in B$ $(1 \leq i \leq n)$ generate the unit ideal, and suppose $A_{g_i}$ is a finite $B_{g_i}$-module. Then $A$ is a finite $B$-module. The proof is identical to (Ex. 3.3c), so we omit this part.
\end{proof}

\item[\textbf{6.}] Let $X$ be an integral scheme. Show that the local ring $\fO_\xi$ of the generic point $\xi$ of $X$ is a field. It is called the \textit{function field} of $X$, and is denoted by $K(X)$. Show also that if $U = \spec{A}$ is any open affine subset of $X$, then $K(X)$ is isomorphic to the quotient field of $A$.

\begin{proof}
    Any nonempty open set $U$ of $X$ must contain $\xi$ since $X - U$ is a proper closed subset of $X$. In particular, any element of $\fO_\xi$ can be represented as a pair $(\spec{A}, f)$ where $\spec{A}$ is an open affine set in $X$ and $f \in A$. We further assume $\spec{A}$ is connected, so $A$ is an integral domain. If $f$ is a nonzero element of $A$, then $(\spec{A}, f)$ is a nonzero element of $\fO_\xi$ with inverse $(\spec{A}, f)^{-1} = (\spec{A_f}, 1/f)$, hence $\fO_\xi$ is a field. Lastly, the distinguished open sets $\spec{A_f}$ form a neighborhood basis of $\xi$, so any element of $\fO_\xi$ can be written as $(\spec{A_f}, a / f^n)$, which is the quotient field of $A$.
\end{proof}

\item[\textbf{7.}] A morphism $f : X \to Y$, with $Y$ irreducible, is \textit{generically finite} if $f^{-1}(\eta)$ is a finite set, where $\eta$ is the generic point of $Y$. A morphism $f : X \to Y$ is \textit{dominant} if $f(X)$ is dense in $Y$. Now let $f : X \to Y$ be a dominant, generically finite morphism of finite type of integral schemes. Show that there is an open dense subset $U \subseteq Y$ such that the induced morphism $f^{-1}(U) \to U$ is finite.

\item[\textbf{8.}] \textit{Normalization.} A scheme is \textit{normal} if all of its local rings are integrally closed domains. Let $X$ be an integral scheme. For each open affine subset $U = \spec{A}$ of $X$, let $\tilde{A}$ be the integral closure of $A$ in its quotient field, and let $\tilde{U} = \spec{\tilde{A}}$. Show that one can glue the schemes $\tilde{U}$ to obtain a normal integral scheme $\tilde{X}$, called the \textit{normalization} of $X$. Show also that there is a morphism $\tilde{X} \to X$, having the following universal property: for every normal integral scheme $Z$, and for every dominant morphisms $f : Z \to X$, $f$ factors uniquely through $\tilde{X}$. If $X$ is of finite type over a field $k$, then the morphism $\tilde{X} \to X$ is a finite morphism.

\begin{lemma}
    If $f : Z \to X$ is a dominant morphism of schemes with $Z$ reduced, then $f^\# : \fO_X \to f_* \fO_Z$ is injective.    
\end{lemma}

\begin{proof}
    Let $U$ be any open subset of $X$. We want to show if $g \in \Gamma(U, \fO_X)$ such that $f^\#(g) = 0 \in \Gamma(f^{-1}(U), \fO_Z)$, then $g = 0$. By lemma 1, we have $f^{-1}(U_g) = (f^{-1}(U))_{f^\#(g)}$, and if $f^\#(g) = 0$, then $(f^{-1}(U))_{f^\#(g)} = \emptyset$, so $U_g$ must not meet the $f(Z)$. But $f(Z)$ is dense in $X$, and $U_g$ is an open set in $X$ by (Ex. 2.16a), so $U_g = \emptyset$, which implies $g$ is nilpotent (this result corresponds to the algebraic fact that the intersection of all prime ideals of a ring is the nilradical of the ring), hence $g = 0$.
\end{proof}

\begin{proof}
    Let $X = \bigcup \spec{A_i}$ be an open affine covering of $X$ where each $A_i$ is an integral domain. For each $i \neq j$, we have an identification $\varphi_{ij} : U_i \to U_j$, which is an isomorphism of open subschemes $U_i \subseteq\spec{A_i}, U_j \subseteq \spec{A_j}$. By Nike's trick, there exists an open covering of $U_i$ by basic open sets $\spec{(A_i)_{f_k}}$ with $f_k \in A_i$ such that $\varphi_{ij}(\spec{(A_i)_{f_k}}) = \spec{(A_j)_{g_k}}$ with $g_k \in A_j$. For each $i$, let $\tilde{A}_i$ be the integral closure of $A_i$ in its quotient field (note by (Ex. 6), every $A_i$ has the same quotient field), and let $\pi_i : \spec{\tilde{A}_i} \to \spec{A_i}$ be the morphism induced by the inclusion $A_i \hookrightarrow \tilde{A}_i$. We have $\pi_i^{-1}(\spec{(A_i)_{f_k}}) = \spec{(\tilde{A}_i)_{f_k}}$ and $(\tilde{A}_i)_{f_k} \cong (\tilde{A}_j)_{g_k}$, so we can naturally glue open subsets of $X'$ that are of the form $\pi_i^{-1}(U_i)$ using $\varphi_{ij}$ to obtain $\tilde{X}$. The morphism $\pi : \tilde{X} \to X$ is obtained by glueing $\pi_i$ accordingly.

    Now suppose $Z$ is a normal integral scheme, and let $f : Z \to X$ be a dominant morphism of schemes. It is clear we can assume $X = \spec{A}$, where $A$ is an integral domain. Then $f$ induces a ring homomorphism $\varphi : A \to B = \Gamma(Z, \fO_Z)$. Let $\tilde{X} = \spec{\tilde{A}}$ be the normalization of $X$ with associated morphism $\pi : \tilde{X} \to X$ induced by the inclusion homomorphism $\iota : A \to \tilde{A}$, where $\tilde{A}$ is the integral closure of $A$ in its quotient field. We want to show there exists a unique morphism $\tilde{f} : Z \to \tilde{X}$ such that $f = \pi \circ \tilde{f}$. Being integrally closed is a local property (A.M. 5.13), so $B$ is integrally closed (it is automatically an integral domain by definition of an integral scheme). Also, $\varphi$ is injective by lemma 2. Since $\tilde{X}$ is affine, by the bijection in (Ex. 24), we have reduced to proving the following universal property for the integral closure of a domain $A \xrightarrow{\iota} \tilde{A}$: for any injective homomorphism $\varphi : A \to B$ where $B$ is an integrally closed domain, there exists a unique homomorphism $\psi : \tilde{A} \to B$ such that $\varphi = \psi \circ \iota$. Any injective homomorphism between integral domains induces an inclusion of fraction fields, so let $\Phi : \text{Frac}(A) \to \text{Frac}(B)$ be induced by $\varphi$, where $\restr{\Phi}{A} = \varphi$. Note that we have inclusions $A \subseteq \tilde{A} \subseteq \text{Frac}(A)$, thus we claim $\psi = \restr{\Phi}{\tilde{A}} : \tilde{A} \to \text{Frac}(B)$ is the desired ring homomorphism. It suffices to show the image of $\psi$ is contained in $B$. If $f \in \tilde{A}$, then there exists an equation of integral dependence $f^n + a_1 f^{n - 1} + \cdots + a_n = 0$ where $a_i \in A$. Then $\Phi(f)$ has an equation of integral dependence $\Phi(f)^n + \varphi(a_1) \Phi(f)^{n - 1} + \cdots + \varphi(a_n) = 0$, and since $B$ is integrally closed, $\Phi(f)$ must be an element of $B$. Since any other $\psi' : \tilde{A} \to B$ such that $\varphi = \psi' \circ \iota$ must agree with $\Phi$ on $\tilde{A}$, $\psi$ is unique by construction.
\end{proof}

\item[\textbf{10.}] \textit{Fibers of a Morphism.}
\begin{itemize} [leftmargin=0cm]
    \item[(a)] If $f : X \to Y$ is a morphism, and $y \in Y$ a point, show that $\text{sp}(X_y)$ is homeomorphic to $f^{-1}(y)$ with the induced topology.
    \item[(b)] Let $X = \spec{k[s, t] / (s - t^2)}$, let $Y = \spec{k[s]}$, and let $f : X \to Y$ be the morphism defined by sending $s \to s$. If $y \in Y$ is the point $a \in k$ with $a \neq 0$, show that the fiber $X_y$, consists of two points, with residue field $k$. If $y \in Y$ corresponds to $0 \in k$, show that the fiber $X_y$ is a nonreduced one-point scheme. If $\eta$ is the generic point of $Y$, show that $X_\eta$ is a one-point scheme, whose residue field is an extension of degree two of the residue field of $\eta$. (Assume $k$ is algebraically closed.)
\end{itemize}

\begin{proof} $ $ \vspace{0pt}
    \begin{itemize} [leftmargin=0cm]
        \item[(a)] We can assume $Y$ to be affine by taking any open affine neighborhood of $y \in Y$ and restricting $f$ to its preimage in $X$. Also, if $U_i = \spec{A_i}$ is an open affine cover of $X$, then $f^{-1}(y) = \bigcup_i (\spec{A_i} \cap f^{-1}(y))$, and $\text{sp}(X_y) = \bigcup_i \text{sp}(U_i \times_Y k(y))$, so we can also assume  $X$ to be affine. Let $X = \spec{A}$ and $Y = \spec{B}$, then $f$ induces a ring homomorphism $f^\# : B \to A$. A point $y \in Y$ corresponds to a prime ideal $\goth{q}$ in $B$, where $k(y) = B_\goth{q} / \goth{q} B_\goth{q}$. Thus, $f^{-1}(y)$ is the set of all prime ideals $\goth{p}$ in $A$ such that $f^{\#-1}(\goth{p}) = \goth{q}$. Next, we look at $\text{sp}(X_y)$. Notice that $X_y = \spec{A \otimes_B k(y)}$, and $A \otimes_B k(y) = A \otimes_B B_\goth{q} / \goth{q} B_\goth{q} = A_\goth{q} / \goth{q} A_\goth{q}$. The prime ideals of $A_\goth{q} / \goth{q} A_\goth{q}$ correspond to prime ideals of $A$ that contain the image of $\goth{q}$ and does not contain meet the image of $B - \goth{q}$, which are precisely the prime ideals of $A$ such that $f^{\#-1}(\goth{p}) = \goth{q}$.

        \item[(b)] $X_y = \spec{k[t] / (a - t^2)}$, $X_\eta = \spec{k(s)[\sqrt{s}]}$. Any element of $k(s)[\sqrt{s}]$ can be written as $F + G\sqrt{s}$, where $F, G \in k(s)$. It is a field, since $(F + G\sqrt{s})^{-1} = (F - G\sqrt{s}) / (F^2 - sG^2)$.
    \end{itemize}
\end{proof}

\item[\textbf{11.}] \textit{Closed Subschemes.}
\begin{itemize} [leftmargin=0cm]
    \item[(a)] Closed immersions are stable under base extension: if $f : Y \to X$ is a closed immersion, and if $X' \to X$ is any morphism, then $f' : Y \times_X X' \to X'$ is also a closed immersion.

    \item[(b)] If $Y$ is a closed subscheme of an affine scheme $X = \spec{A}$, then $Y$ is also affine, and in fact $Y$ is the closed subscheme determined by a suitable ideal $\goth{a} \subseteq A$ as the image of the closed immersion $\spec{A / \goth{a}} \to \spec{A}$.

    \item[(c)] Let $Y$ be a closed subset of a scheme $X$, and give $Y$ the reduced induced subscheme structure. If $Y'$ is any other closed subscheme of $X$ with the same underlying topological space, show that the closed immersion $Y \to X$ factors through $Y'$. We express this property by saying that the reduced induced structure is the smallest subscheme structure on a closed subset.

    \item[(d)] Let $f : Z \to X$ be a morphism. Then there is a unique closed subscheme $Y$ of $X$ with the following property: the morphism $f$ factors through $Y$, and if $Y'$ is any other closed subscheme of $X$ through which $f$ factors, then $Y \to X$ factors through $Y'$ also. We call $Y$ the \textit{scheme-theoretic image} of $f$. If $Z$ is a reduced scheme, then $Y$ is just the reduced induced structure on the closure of the image $f(Z)$.
\end{itemize}

\begin{proof} $ $ \vspace{0pt}
   \begin{itemize} [leftmargin=0cm]
    \item[(a)] Consider the special case when $X = \spec{A}, Y = \spec{A / \goth{a}}$, and $X' = \spec{B}$ where $\goth{a}$ is an ideal of $A$ and $B$ is any $A$-algebra. The natural map $A \to A / \goth{a}$ induces a closed immersion $Y \to X$, and the structure homomorphism $A \to B$ induces a morphism of schemes $X' \to X$. The fiber product $Y \times_X X'$ is equal to the spectra of the tensor product $A / \goth{a} \otimes_A B$, which is isomorphic to $B / \goth{a}B$. The induced structure morphism $Y \times_X X' \to X'$ corresponds to the canonical homomorphism $B \to B / \goth{a} B$, thus $Y \times_X X' \to X'$ is a closed immersion. In other words, this property of closed immersions corresponds to the algebraic fact that the tensor operation is a right exact functor. In the general case, we can still assume $X$ to be affine by taking an open affine cover of $X$. Then $Y$ is an affine scheme by part (b), and if $U$ is any open subset of $X'$, $f^{'-1}(U) = Y \times_X U$, so by taking an open affine cover of $X'$, we can reduce to the case when $X'$ is affine, which is just the special case as in above.

    \item[(b)] Let $Y$ be a closed subscheme of an affine scheme $X = \spec{A}$, and let $\varphi : A \to B = \Gamma(Y, \fO_Y)$ be the induced ring homomorphism of global sections. Fix $y \in Y$, let $V = \spec{C}$ be an open affine neighborhood of $y$ as a subspace of $Y$, and let $\rho : B \to C$ be the restriction homomorphism. By definition of the subspace topology, there exists an open set $U$ of $X$ such that $V = U \cap Y$. We can cover $U$ by distinguished open sets, and at least one of them must contain $y$, so let $X_f$ be such set with $f \in A$. By lemma 1, $X_f \cap Y = Y_{\varphi(f)} = \spec{C_{\rho(\varphi(f))}}$. Thus, we can cover $Y$ by open affines of the form $X_{f_i} \cap Y$. Since $Y$ is homeomorphic to a closed subset of a quasi-compact set, it is also quasi-compact, so a finite number will do. By adding some more $f_i$ with $D(f_i) \cap Y = \emptyset$ by taking an open cover of $X - Y$, we assume $X_{f_i}$ cover $X$. Such collection of $X_{f_i}$ cover $X$ if and only if $f_i$ generate the unit ideal in $A$, so $\varphi(f_i)$ must generate the unit ideal of $B$. Hence, $Y$ is affine by (Ex. 2.17b). The quotient ring $A / \ker{\varphi}$ is a subring of $B$, so the affine scheme $X' = \spec{A / \ker{\varphi}}$ contains $Y$ as a dense subset. $X'$ is also homeomorphic to a closed subset of $X$, hence $Y = \spec{A / \ker{\varphi}}$.

    \item[(c)] By taking an affine cover of $X$, we reduce to the case when $X = \spec{A}$ is affine. Let $\goth{a}$ be the ideal in $A$ that corresponds to the reduced induced structured of $Y$. By part (b), there exists an ideal $\goth{b}$ in $A$ such that $Y' = \spec{A / \goth{b}}$. A morphism of schemes $Y \to Y'$ corresponds to a ring homomorphism $A / \goth{b} \to A / \goth{a}$. Recall that $\goth{a}$ is the largest ideal of $A$ such that $V(\goth{a}) = \text{sp}(Y) = \text{sp}(Y')$; in particular, $\goth{b} \subseteq \goth{a}$, so the natural projection map $A / \goth{b} \to A / \goth{a}$ is well-defined and is the desired ring homomorphism.

    \item[(d)] Again, reduce to the affine case. A morphism of affine schemes $f : \spec{B} \to \spec{A}$ correspond to a ring homomorphism $\varphi : A \to B$. Then $Y = \spec{A / \ker{\varphi}}$ is the desired closed subscheme. If $Z$ is reduced, then $B$ is a reduced ring. Then $\ker{\varphi}$ contains the nilradical of $A$, so $A / \ker{\varphi}$ is reduced, hence the intersection of all prime ideals of $A / \ker{\varphi}$ is the zero ideal.
   \end{itemize} 
\end{proof}

\item[\textbf{13.}] \textit{Properties of Morphisms of Finite Type.}
\begin{itemize}
    \item[(a)] A closed immersion is a morphism of finite type.
    \item[(b)] A quasi-compact open immersion is of finite type.
    \item[(c)] A composition of two morphisms of finite type is of finite type.
    \item[(d)] Morphisms of finite type are stable under base extension.
    \item[(e)] If $X$ and $Y$ are schemes of finite type over $S$, then $X \times_S Y$ is of finite type over $S$.
    \item[(f)] If $X \xrightarrow{f} Y \xrightarrow{g} Z$ are two morphisms, and if $f$ is quasi-compact, and $g \circ f$ is of finite type, then $f$ is of finite type.
    \item[(g)] If $f : X \to Y$ is a morphism of finite type, and if $Y$ is noetherian, then $X$ is noetherian.
\end{itemize}

\begin{proof} $ $ \vspace{0pt}
   \begin{itemize} [leftmargin=0cm]
    \item[(a)] Let $f : Y \to X$ be a closed immersion. By abuse of notation, also denote $Y$ as a closed subset of $\text{sp}(X)$. If $U$ is quasi-compact in $X$, then $f^{-1}(U) = U \cap Y$, which is quasi-compact since any closed subset of a quasi-compact set is also quasi-compact, so $f$ is a quasi-compact morphism. Also, if $U = \spec{A}$ is an open affine subset of $X$, then $f^{-1}(U) = U \cap Y$ is a closed subscheme of $U$ equal to the spectra of $A/ \goth{a}$ for some ideal $\goth{a}$ of $A$ by (Ex. 3.11b), which is a finitely generated $A$-algebra, so $f$ is also locally of finite type. Hence, $f$ is of finite type.

    \item[(b)] Let $f : Y \to X$ be an open immersion. If $U = \spec{A}$ is any open affine set in $X$, then $f^{-1}(U) = U \cap Y$ can be covered by distinguished open sets of $U$, which are spectra of $A_f$ for some $f \in A$. Any such $A_f$ is a finitely generated $A$-algebra. Hence, an open immersion is locally of finite type, so a quasi-compact open immersion is of finite type by (Ex. 3.3a).

    \item[(c)] Let $f : X \to Y$ and $g : Y \to Z$ be morphisms of finite type. Let $W = \spec{C}$ be an open affine subset of $Z$. Since $g$ is of finite type, $g^{-1}(W)$ can be covered by a finite number of open affines $V_i = \spec{B_i}$, where each $B_i$ is a finitely generated $C$-algebra, and similarly each $f^{-1}(V_i)$ can be covered by a finite number of open affines $U_{ij} = \spec{A_{ij}}$, where each $A_{ij}$ is a finitely generated $B_i$-algebra. Each $A_{ij}$ is also a finitely-generated $A$-algebra since $B_i$ is a finitely generated $A$-algebra, and there are a finite number of $U_{ij}$ with $(g \circ f)^{-1}(W) = f^{-1}(g^{-1}(W)) = \bigcup U_{ij}$, $g \circ f$ is a morphism of finite type.

    \item[(d)] This corresponds to the algebraic fact that if $A$ is a finitely generated $B$-algebra and $C$ is any other $B$-algebra, then $A \otimes_A C$ is a finitely generated $C$-algebra.

    \item[(e)] This corresponds to the algebraic fact that the tensor product of two finitely generated algebras is also finitely generated.

    \item[(f)] By (Ex. 3.3a), it suffices to show $f$ is locally of finite type. Let $W = \spec{C}$ be an open affine subset of $Z$, and let $V_i = \spec{B_i}$ be an open affine cover of $g^{-1}(W)$. By (Ex. 3.3c), for each $i$, we can cover $f^{-1}(V_i)$ by open affines $U_{ij} = \spec{A_{ij}}$ where each $A_{ij}$ is a finitely generated $C$-algebra. We have a composition of structure morphisms $C \to B_i \to A_{ij}$, where $A_{ij}$ is finitely generated over the image of $C$ in $A_{ij}$. Since the image of $C$ in $A_{ij}$ is contained in the image of $B_i$ in $A_{ij}$, $A_{ij}$ is also a finitely generated $B_i$-algebra.

    \item[(g)] This corresponds to the fact that any finitely generated algebra over a noetherian ring is also noetherian as a ring.
   \end{itemize} 
\end{proof}

\item[\textbf{16.}] \textit{Noetherian Induction.} Let $X$ be a noetherian topological space, and let $\mathscr{P}$ be a property of closed subsets of $X$. Assume that for any closed subset $Y$ of $X$, if $\mathscr{P}$ holds for every proper closed subset of $Y$, then $\mathscr{P}$ holds for $Y$. (In particular, $\mathscr{P}$ must hold for the empty set.) Then $\mathscr{P}$ holds for $X$.

\begin{proof}
    Suppose $\mathscr{P}$ does not hold for $X$, and set $X_1 = X$. Recursively define $X_{n + 1}$ to be any proper closed subset of $X_{n}$ such that $\mathscr{P}$ does not hold. This is a nonterminating decreasing sequence of closed subsets of $X$ by construction, a contradiction.
\end{proof}

\item[\textbf{20.}] \textit{Dimension.} Let $X$ be an integral scheme of finite type over a field $k$ (not necessarily algebraically closed). Use appropriate results from (I,\textsection 1) to prove the following.
\begin{itemize}
    \item[(a)] For any closed point $P \in X$, $\dim{X} = \dim{\fO_P}$, where for rings, we always mean the Krull dimension.
    \item[(b)] Let $K(X)$ be the function field of $X$. Then $\dim{X} = \text{tr.d.} K(X) / k$.
    \item[(c)] If $Y$ is a closed subset of $X$, then $\text{codim}(Y, X) = \inf\{\dim{\fO_{P, X}} \mid P \in Y\}$.
    \item[(d)] If $Y$ is a closed subset of $X$, then $\dim{Y} + \text{codim}(Y, X) = \dim{X}$.
    \item[(e)] If $U$ is a nonempty open subset of $X$, then $\dim{U} = \dim{X}$.
    \item[(f)] If $k \subseteq k'$ is a field extension, then every irreducible component of $X' = X \times_k k'$ has $\text{dimension} = \dim{X}$.
\end{itemize}

\begin{proof} $ $ \vspace{0pt}
\begin{itemize} [leftmargin=0cm]
    \item[(a)] Let $V$ be a variety over $k$. The bijection between the open sets of $V$ and open sets of $t(V)$ induced by the map $\alpha$ in the proof of (2.6) implies $\dim{V} = \dim{t(V)}$.
    If $X$ is an integral scheme of finite type over a field $k$, then we can cover $X$ by a finite number of open affines $U_i = \spec{A_i}$, where each $A_i$ is of the form $k[x_1, \dots, x_{r_i}] / \goth{p}_i$ for some $r_i > 0$ and prime ideal $\goth{p}_i$ in $k[x_1, \dots, x_{r_i}]$. Then $P \in U_i$ for some $i$, and if $P$ is closed, then it corresponds to a maximal ideal $\goth{m}_P$ in $A_i$, so by (A.M. 11.25) and (I, 1.7), $\dim{\fO_P} = \dim{A_i} = \dim{U_i}$. 
    Also, $\dim{U_i} = \dim{X}$ for all $i$ by (I, Ex. 1.10b) and from the fact that $\dim{U_i} = \dim{U_j}$.
    Indeed, $X$ itself is irreducible since it is an integral scheme and therefore has a unique generic point. Then $U_i \cap U_j \neq \emptyset$ for any $i, j$, so for all $P \in U_i \cap U_j$, $\dim{U_i} = \dim{\fO_{U_i, P}} = \dim{\fO_{U_j, P}} = \dim{U_j}$. Hence, $\dim{X} = \dim{\fO_P}$ for all closed $P \in X$.

    \item[(b)] By part (a), (II, Ex. 3.6), (I, 1.8A), $\trd{K(X)} / k = \dim{U_i} = \dim{X}$.

    \item[(c)] Let $X = \spec{A}$, where $A$ is an integral domain that is finitely generated over a field $k$, and let $Y$ be an irreducible, closed subset of $X$.
    The closed subscheme $Y$ corresponds to a prime ideal $\goth{p}$ in $A$, and any irreducible closed set containing $Y$ corresponds to a prime ideal contained in $\goth{p}$, so $\codim{Y}{X}$ equals to the height of $\goth{p}$. Points in $Y$ correspond to prime ideals containing $\goth{p}$, so the infimum of the dimension of all local rings $\fO_{P, X}$ over $P \in Y$ equals to the infimum of the height of all prime ideals of $A$ which contain $\goth{p}$, which is just the height of $\goth{p}$. If $Y$ is any closed subset, then $Y$ corresponds to an ideal $\goth{a}$ in $A$, and any irreducible closed set of $Y$ contained in $Y$ corresponds to a prime ideal of $A$ that contains $\goth{a}$. From the case when $Y$ is irreducible, the infimum of the codimension of all irreducible closed sets contained in $Y$ equals to the infimum of the height of prime ideals containing $\goth{a}$. The height of such prime ideals is precisely the dimension of the local ring of points $P \in Y$. If $X$ is any integral scheme of finite type over a field $k$, then we can cover $X$ by a finite number of open affines $U_i = \spec{A_i}$ where $\dim{A_i} = \dim{A_j}$. If $P \in U_i \cap U_j \cap Y$, then $\dim{\fO_{P, X}} = \dim{\fO_{P, U_i}} = \dim{\fO_{P, U_j}}$; in particular, the prime ideals that $P$ corresponds to in $A_i, A_j$ have the same height. Thus, we can just reduce to the affine case.
    
    \item[(d)] By the same reason as part (c), we can reduce to the affine case, which follows from (\S 1, 1.7) and (\S 1, 1.8Ab). 

    \item[(e)] By (\S 1, Ex. 1.10b), $\dim{U} = \sup{\dim{U \cap U_i}} = \dim{U_i} = \dim{X}$. 

    \item[(f)] If $U_i = \spec{A_i}$ is an open affine cover of $X$, then $U_i' = U_i \times_k k' = \spec{A_i \otimes_k k'}$ is an open affine cover of $X$, so by (\S 1, Ex. 1.10b) and part (a), we reduce to the case when $X = \spec{A}$ is affine. Suppose $A = k[x_1, \dots, x_n] / \goth{p}$ for some prime ideal $\goth{p}$. Then $X'$ is the spectra of the ring $k'[x_1, \dots, x_n]/\goth{p}'$, where $\goth{p}'$ is the extension of $\goth{p}$ in $k'[x_1, \dots, x_n]$. The irreducible components of $X'$ correspond to minimal prime ideals of $\goth{p}'$. Let $\goth{q}$ be a minimal prime ideal of $\goth{p}'$ so that $\goth{p} = \goth{q} \cap k[x_1, \dots, x_n]$. We want to show $\dim{k[x_1, \dots, x_n] / \goth{p}} = \dim{k'[x_1, \dots, x_n] / \goth{q}}$. The case when $k'$ is an algebraic extension of $k$ is immediate from the going-up going-down theorems, so assume $k$ and $k'$ are algebraically closed. Let $K$, $K'$ be the fraction fields of $k[x_1, \dots, x_n] / \goth{p}$, $k'[x_1, \dots, x_n] / \goth{q}$, then by part (b) or (\S 1, 1.8Aa) it suffices to show $\text{tr.d.} K/k = \text{tr.d.} K'/k'$. Since $k, k'$ are algebraically closed, $K, K'$ are purely transcendental extensions, so we can write $K = k(y_1, \dots, y_r), K' = k'(y'_1, \dots, y'_{r'})$, where $r = \text{tr.d.}K/k, r' = \text{tr.d.}K'/k'$. By Noether's normalization lemma, we can choose $y_i$ to be linear combinations of $x_1, \dots, x_n$, so after a suitable $k$-algebra automorphism, we can assume $\goth{p} = (x_1, \dots, x_{n - r})$ so that $y_i = x_i$ for $i = n - r + 1, \dots, n$. Then the extension of $\goth{p}$ in $k'[x_1, \dots, x_n]$ is also a prime ideal generated by $x_1, \dots, x_{n - r}$, which implies $r = r'$.
    
    Since $n = \dim{k[x_1, \dots, x_n]} = \dim{k'[x_1, \dots, x_n]}$, by (\S 1, 1.8Ab) we have the following corollary:
    \begin{corollary}
        Let $k'/k$ be any field extension. If $\goth{p}$ is a prime ideal in $k[x_1, \dots, x_n]$, then any minimal prime ideal of the extension of $\goth{p}$ in $k'[x_1, \dots, x_n]$ has the same height as $\goth{p}$.
    \end{corollary}
\end{itemize}
\end{proof}

\end{enumerate}

\end{document}

% \item[\textbf{15.}] Let $X$ be a scheme of finite type over a field $k$ (not necessarily algebraically closed).
% \begin{itemize} [leftmargin=0em]
%     \item[(a)] Show that the following three conditions are equivalent (in which case we say that $X$ is \textit{geometrically irreducible}).
%     \begin{itemize}
%         \item[(i)] $X \times_k \overline{k}$ is irreducible, where $\overline{k}$ denotes the algebraic closure of $k$. (By abuse of notation, we write $X \times_k \overline{k}$ to denote $X \times_{\spec{k}} \spec{\overline{k}}$.)
%         \item[(ii)] $X \times_k k_s$ is irreducible, where $k_s$ denotes the separable closure of $k$.
%         \item[(iii)] $X \times_k K$ is irreducible for every extension field $K$ of $k$.
%     \end{itemize}
%     \item[(b)] Show that the following three conditions are equivalent (in which case we say $X$ is \textit{geometrically reduced}.)
%     \begin{itemize}
%         \item[(i)] $X \times_k \overline{k}$ is reduced.
%         \item[(ii)] $X \times_k k_p$ is reduced, where $k_p$ denotes the perfect closure of $k$.
%         \item[(iii)] $X \times_k K$ is reduced for all extension field $K$ of $k$.
%     \end{itemize}
%     \item[(c)] We say that $X$ is \textit{geometrically integral} if $X \times_k \overline{k}$ is integral. Give examples of integral schemes which are neither geometrically irreducible nor geometrically reduced.
% \end{itemize}

% \item[\textbf{17.}] \textit{Zariski Spaces.} A topological space $X $ is a \textit{Zariski space} if it is noetherian and every (nonempty) closed irreducible subset has a unique generic point.
% For example, let $R$ be a discrete valuation ring, and let $T = \text{sp}(\spec{R})$. Then $T$ consists of two points $t_0 = \text{the maximal ideal}$, $t_1 = \text{zero ideal}$. The open subsets are $\emptyset, \{ t_1 \}, \text{ and } T$. This is an irreducible Zariski space with generic point $t_1$.
% \begin{itemize} [leftmargin=0em]
%     \item[(a)] Show that if $X$ is a noetherian scheme, then $\text{sp}(X)$ is a Zariski space.
%     \item[(b)] Show that any minimal nonempty closed subset of a Zariski space consists of one point. We call these \textit{closed points.}
%     \item[(c)] Show that a Zariski space $X$ satisfies the axiom $T_0$: given any two distinct points of $X$, there is an open set consisting one but not the other.
%     \item[(d)] If $X$ is an irreducible Zariski space, then its generic point is contained in every nonempty open subset of $X$.
%     \item[(e)] If $x_0, x_1$ are points of a topological space $X$, and if $x_0 \in \overline{\{x_1 \}}$, then we say that $x_1$ \textit{specializes} to $x_0$, written $x_1 \rightsquigarrow x_0$. We also say $x_0$ is a \textit{specialization} of $x_1$, or that $x_1$ is a \textit{generization} of $x_0$. Now let $X$ be a Zariski space. Show that the minimal points, for the partial ordering determined by $x_1 > x_0$ if $x_1 \rightsquigarrow x_0$, are closed points, and the maximal points are the generic points of that irreducible components of $X$. Show also that a closed subset contains every specialization of any of its points. Similarly, open subsets are \textit{stable under generization.}
%     \item[(f)] Let $t$ be the functor on topological spaces introduced in the proof of (2.6). If $X$ is a noetherian topological space, show that $t(X)$ is a Zariski space. Furthermore $X$ itself is a Zariski space if and only if the map $\alpha : X \to t(X)$ is a homeomorphism.
% \end{itemize}

% \item[\textbf{18.}] \textit{Constructible Sets.} Let $X$ be a Zariski topological space. A \textit{constructible subset} of $X$ is a subset which belongs to the smallest family $\mathfrak{F}$ of subsets such that
% \begin{itemize}
%     \item[(1)] every open subset is in $\mathfrak{F}$,
%     \item[(2)] a finite intersection of elements of $\mathfrak{F}$ is in $\mathfrak{F}$,
%     \item[(3)] and the complement of an element of $\mathfrak{F}$ is in $\mathfrak{F}$.
% \end{itemize}
% \begin{itemize} [leftmargin=0cm]
%     \item[(a)] A subset of $X$ is \textit{locally closed} if it is the intersection of an open subset with a closed subset. Show that a subset of $X$ is constructible if and only if it can be written as a finite disjoint union of locally closed subsets.
%     \item[(b)] Show that a constructible subset of an irreducible Zariski space $X$ is dense if and only if it contains the generic point. Furthermore, in that case it contains a nonempty open subset.
%     \item[(c)] A subset $S$ of $X$ is closed if and only if it is constructible and stable under specialization. Similarly, a subset $T$ of $X$ is open if and only if it is constructible and stable under generization.
% \end{itemize}

% \item[\textbf{19.}] Let $f : X \to Y$ be a morphism of finite type of noetherian schemes. Then the image of any constructible subset of $X$ is a constructible subset of $Y$. In particular, $f(X)$, which need not be either open or closed, is a constructible subset of $Y$. Prove this theorem in the following steps:
% \begin{itemize} [leftmargin=0cm]
%     \item[(a)] Reduce to showing that $f(X)$ itself is constructible, in the case where $X$ and $Y$ are affine, integral noetherian schemes, and $f$ is a dominant morphism.
%     \item[(b)] In that case, show that $f(X)$ contains a nonempty open subset of $Y$ by using the following result from commutative algebra: let $A \subseteq B$ be an inclusion of noetherian integral domains, such that $B$ is a finitely generated $A$-algebra. Then given a nonzero element $b \in B$, there is a nonzero element $a \in A$ with the following property: if $\varphi : A \to K$ is any homomorphism of $A$ to an algebraically closed field $K$, such that $\varphi(a) \neq 0$, then $\varphi$ extends to a homomorphism $\varphi'$ of $B$ into $K$, such that $\varphi'(b) \neq 0$.
%     \item[(c)] Now use noetherian induction on $Y$ to complete the proof.
%     \item[(d)] Give some examples of morphisms $f : X \to Y$ of varieties over an algebraically closed field $k$, to show that $f(X)$ need not be either open or closed.
% \end{itemize}

% \item[\textbf{22.}] \textit{Dimension of Fibers of a Morphism.} Let $f : X \to Y$ be a dominant morphism of integral schemes of finite type over a field $k$.
% \begin{itemize} [leftmargin=0cm]
%     \item[(a)] Let $Y'$ be a closed irreducible subset of $Y$, whose generic point $\eta'$ is contained in $f(X)$. Let $Z$ be any irreducible component of $f^{-1}(Y)$, such that $\eta' \in f(Z)$, and show that $\text{codim}(Z, X) \leq \text{codim}(Y', Y)$.
%     \item[(b)] Let $e = \dim{X} - \dim{Y}$ be the \textit{relative dimension} of $X$ over $Y$. For any point $y \in f(X)$, show that every irreducible component of the fiber $X_y$ has dimension $\geq e$. 
%     \item[(c)] Show that there is a dense open subset $U \subseteq X$, such that for any $y \in f(U)$, $\dim{U_y} = e$.
%     \item[(d)] Going back to our original morphism $f : X \to Y$, for any integer $h$, let $E_h$ be the set of points $x \in X$ such that, letting $y = f(x)$, there is an irreducible component $Z$ of the fiber $X_y$, containing $x$, and having $\dim{Z} > h$. Show that
%     \begin{itemize}
%         \item[(1)] $E_e = X$;
%         \item[(2)] if $h > e$, then $E_h$ is not dense in $X$;
%         \item[(3)] $E_h$ is closed, for all $h$.
%     \end{itemize}
%     \item[(e)] Prove the following theorem of Chevalley - See Cartan and Chevalley [1, exposé 7]. For each integer $h$, let $C_h$ be the set of points $y \in Y$ such that $\dim{X_y} = h$. Then the subsets $C_h$ are constructible, and $C_e$ contains an open dense subset of $Y$.
% \end{itemize}
