\documentclass{article}
\usepackage[margin=0.75in]{geometry}
\usepackage{amsmath}
\usepackage{amsthm}
\usepackage{amssymb}
\usepackage{enumitem}
\usepackage{tikz-cd}
\usepackage{yfonts}
\usepackage{mathrsfs}
\usepackage{mathabx}
\DeclareMathAlphabet{\mathpzc}{OT1}{pzc}{m}{it}
\newcommand{\goth}[1]{\mathfrak{#1}}
\newcommand{\fF}{\mathscr{F}}
\newcommand{\fG}{\mathscr{G}}
\newcommand{\fE}{\mathscr{E}}
\newcommand{\fO}{\mathscr{O}}
\newcommand{\fL}{\mathscr{L}}
\newcommand{\fM}{\mathscr{M}}
\newcommand{\fI}{\mathscr{I}}
\newcommand{\fT}{\mathscr{T}}
\newcommand{\fK}{\mathscr{K}}
\newcommand{\fS}{\mathscr{S}}
\newcommand{\fJ}{\mathscr{J}}
\newcommand{\fR}{\mathscr{R}}
\newcommand{\fH}{\mathscr{H}}
\newcommand{\PP}{\mathbb{P}}
\newcommand{\gm}{\goth{m}}
\newcommand{\A}{\mathbb{A}}
\newcommand{\R}{\mathbb{R}}
\newcommand{\C}{\mathbb{C}}
\newcommand{\Q}{\mathbb{Q}}
\newcommand{\N}{\mathbb{N}}
\newcommand{\Z}{\mathbb{Z}}
\newtheorem{theorem}{Theorem}
\newtheorem{lemma}{Lemma}
\newtheorem{corollary}{Corollary}
\DeclareMathOperator{\id}{id}
\DeclareMathOperator{\bProj}{\mathpzc{Proj}}
\DeclareMathOperator{\Frac}{Frac}
\DeclareMathOperator{\rk}{rank}
\DeclareMathOperator{\pic}{Pic}
\DeclareMathOperator{\cacl}{CaCl}
\DeclareMathOperator{\trd}{tr.d.}
\DeclareMathOperator{\cl}{Cl}
\DeclareMathOperator{\Div}{Div}
\DeclareMathOperator{\coker}{coker}
\DeclareMathOperator{\len}{length}
\DeclareMathOperator{\height}{height}
\DeclareMathOperator{\supp}{Supp}
\DeclareMathOperator{\proj}{Proj}
\DeclareMathOperator{\im}{im}
\DeclareMathOperator{\Hom}{Hom}
\DeclareMathOperator{\Der}{Der}
\DeclareMathOperator{\spec}{Spec}
\newcommand\srestr[2]{{
  \left.\kern-\nulldelimiterspace % automatically resize the bar with \right
  #1 % the function
  \vphantom{\small|} % pretend it's a little taller at normal size
  \right|_{#2} % this is the delimiter
}}
\newcommand\restr[2]{{% we make the whole thing an ordinary symbol
  \left.\kern-\nulldelimiterspace % automatically resize the bar with \right
  #1 % the function
  \vphantom{\big|} % pretend it's a little taller at normal size
  \right|_{#2} % this is the delimiter
}}

% 2.4, 2.6, 2.7
% 3.1, 3.2, 3.6, 3.7
% 4.1, 4.2, 4.3, 4.4, 4.5
% 5.1, 5.2, 5.3, 5.10
% 6.1, 6.3, 6.6, 6.7
% 7.1, 7.3
% 8.1, 8.2, 8.3, 
% 9.3, 9.4, 9.11
% 10.1, 10.2, 10.3, 10.5, 10.6
% 11.1, 11.2, 11.8
% 12.1, 12.2

\title{Chapter 3, Section 8}

\usepackage{xcolor}

\pagecolor[RGB]{8,27,31}

\color[RGB]{255,255,255}

\begin{document}
\maketitle
\begin{enumerate} [label=\textbf{\arabic*.}, leftmargin=0em]

\item Let $f : X \to Y$ be a continuous map of topological spaces. Let $\fF$ be a sheaf of Abelian groups on $X$, and assume that $R^i f_*(\fF) = 0$ for all $i > 0$. Show that there are natural isomorphisms, for each $i \geq 0$,
\begin{equation*}
  H^i(X, \fF) \cong H^i(Y, f_* \fF).
\end{equation*}
(This is a degenerated case of the Leray spectral sequence.)

\item Let $f : X \to Y$ be an affine morphism of schemes (II, Ex. 5.17) with $X$ Noetherian, and let $\fF$ be a quasi-coherent sheaf on $X$. Show that the hypotheses of (Ex. 8.1) are satisfies, and hence that $H^i(X, \fF) \cong H^i(Y, f_* \fF)$ for each $i \geq 0$. (This gives another proof of (Ex. 4.1).)

\item Let $f : X \to Y$ be a morphism of ringed spaces, let $\fF$ be an $\fO_X$-module, and let $\fE$ be a locally free $\fO_Y$-module of finite rank. Prove the \textit{projection formula} (cf. (II, Ex. 5.1))
\begin{equation*}
  R^if_*(\fF \otimes f^* \mathscr{E}) \cong R^i f_*(\fF) \otimes \mathscr{E}.
\end{equation*}

\item Let $Y$ be a Noetherian scheme, and let $\mathscr{E}$ be a locally free $\fO_Y$-module of rank $n + 1$, $n \geq 1$. Let $X = \PP(\mathscr{E})$ (II, \S 7), with the invertible sheaf $\fO_X(1)$ and the projection morphism $\pi : X \to Y$.
\begin{itemize}
  \item[(a)] Then $\pi_*(\fO(l)) \cong S^l(\fE)$ for $l \geq 0$, $\pi_*(\fO(l)) = 0$ for $l < 0$ (II, 7.11); $R^i\pi_*(\fO(l)) = 0$ for $0 < i < n$ and all $l \in \Z$; and $R^n \pi_*(\fO(l)) = 0$ for $l > - n - 1$.
  \item[(b)] Show there is a natural exact sequence
  \begin{equation*}
    0 \to \Omega_{X / Y} \to (\pi^* \fE)(-1) \to \fO \to 0,
  \end{equation*}
  cf. (II, 8.13), and conclude that the \textit{relative canonical sheaf} $\omega_{X/Y} = \bigwedge^n \Omega_{X / Y}$ is isomorphic to $(\pi^* \bigwedge^{n + 1} \fE)(-n - 1)$. Show furthermore that there is a natural isomorphism $R^n \pi_*(\omega_{X / Y}) \cong \fO_Y$ (cf. (7.1.1)).
  \item[(c)] Now show, for any $l \in \Z$, that
  \begin{equation*}
    R^n \pi_*(\fO(l)) \cong \pi_*(\widecheck{\fO(-l - n - 1)}) \otimes \widecheck{\bigwedge^{n + 1} \fE}
  \end{equation*}
  \item[(d)] Show that $p_a(X) = (-1)^n p_a(Y)$ (use (Ex. 8.1)) and $p_g(X) = 0$ (use II, 8.11).
  \item[(e)] In particular, if $Y$ is a nonsingular projective curve of genus $g$, and $\fE$ a locally free sheaf of rank $2$, then $X$ is a projective surface with $p_a = -g$, $p_g = 0$, and irregularity $g$ (7.12.3). This kind of surface is called a \textit{geometrically ruled surface} (V, \S 2).
\end{itemize}

\end{enumerate}
\end{document}
