\documentclass{article}
\usepackage[margin=0.75in]{geometry}
\usepackage{amsmath}
\usepackage{amsthm}
\usepackage{amssymb}
\usepackage{enumitem}
\usepackage{tikz-cd}
\usepackage{yfonts}
\usepackage{mathrsfs}
\DeclareMathAlphabet{\mathpzc}{OT1}{pzc}{m}{it}
\newcommand{\goth}[1]{\mathfrak{#1}}
\newcommand{\fF}{\mathscr{F}}
\newcommand{\fG}{\mathscr{G}}
\newcommand{\fE}{\mathscr{E}}
\newcommand{\fO}{\mathscr{O}}
\newcommand{\fL}{\mathscr{L}}
\newcommand{\fM}{\mathscr{M}}
\newcommand{\fI}{\mathscr{I}}
\newcommand{\fT}{\mathscr{T}}
\newcommand{\fK}{\mathscr{K}}
\newcommand{\fS}{\mathscr{S}}
\newcommand{\fJ}{\mathscr{J}}
\newcommand{\fR}{\mathscr{R}}
\newcommand{\fH}{\mathscr{H}}
\newcommand{\fN}{\mathscr{N}}
\newcommand{\PP}{\mathbb{P}}
\newcommand{\gm}{\goth{m}}
\newcommand{\A}{\mathbb{A}}
\newcommand{\R}{\mathbb{R}}
\newcommand{\C}{\mathbb{C}}
\newcommand{\Q}{\mathbb{Q}}
\newcommand{\N}{\mathbb{N}}
\newcommand{\Z}{\mathbb{Z}}
\newtheorem{theorem}{Theorem}
\newtheorem{lemma}{Lemma}
\newtheorem{corollary}{Corollary}
\DeclareMathOperator{\id}{id}
\DeclareMathOperator{\bProj}{\mathpzc{Proj}}
\DeclareMathOperator{\Frac}{Frac}
\DeclareMathOperator{\rk}{rank}
\DeclareMathOperator{\pic}{Pic}
\DeclareMathOperator{\cacl}{CaCl}
\DeclareMathOperator{\trd}{tr.d.}
\DeclareMathOperator{\cl}{Cl}
\DeclareMathOperator{\Div}{Div}
\DeclareMathOperator{\coker}{coker}
\DeclareMathOperator{\len}{length}
\DeclareMathOperator{\height}{height}
\DeclareMathOperator{\supp}{Supp}
\DeclareMathOperator{\proj}{Proj}
\DeclareMathOperator{\im}{im}
\DeclareMathOperator{\Hom}{Hom}
\DeclareMathOperator{\Der}{Der}
\DeclareMathOperator{\spec}{Spec}
\DeclareMathOperator{\rHom}{\mathpzc{Hom}}
\newcommand\srestr[2]{{
  \left.\kern-\nulldelimiterspace % automatically resize the bar with \right
  #1 % the function
  \vphantom{\small|} % pretend it's a little taller at normal size
  \right|_{#2} % this is the delimiter
}}
\newcommand\restr[2]{{% we make the whole thing an ordinary symbol
  \left.\kern-\nulldelimiterspace % automatically resize the bar with \right
  #1 % the function
  \vphantom{\big|} % pretend it's a little taller at normal size
  \right|_{#2} % this is the delimiter
}}

% 2.4, 2.6, 2.7
% 3.1, 3.2, 3.6, 3.7
% 4.1, 4.2, 4.3, 4.4, 4.5
% 5.1, 5.2, 5.3, 5.10
% 6.1, 6.3, 6.6, 6.7
% 7.1, 7.3
% 8.1, 8.2, 8.3, 
% 9.3, 9.4, 9.11
% 10.1, 10.2, 10.3, 10.5, 10.6
% 11.1, 11.2, 11.8
% 12.1, 12.2

\title{Chapter 3, Section 10}

\usepackage{xcolor}

\pagecolor[RGB]{8,27,31}

\color[RGB]{255,255,255}

\begin{document}
\maketitle
\begin{enumerate} [label=\textbf{\arabic*.}, leftmargin=0em]


\item Let $f : X \to Y$ be a property, flat morphism of varieties over $k$. Suppose for some point $y \in Y$ that the fiber $X_y$ is smooth over $k(y)$. Then show that there is an open neighborhood $U$ of $y$ in $Y$ such that $f : f^{-1}(U) \to U$ is smooth.

\item A morphism $f : X \to Y$ of schemes of finite type over $k$ is \textit{étale} if it is smooth of relative dimension $0$. It is \textit{unramified} if for every $x \in X$, letting $y = f(x)$, we have $\goth{m}_y \cdot \fO_x = \goth{m}_x$, and $k(x)$ is a separable algebraic extension of $k(y)$. Show that the following conditions are equivalent:
\begin{itemize}
  \item[(i)] $f$ is étale;
  \item[(ii)] $f$ is flat, and $\Omega_{X/Y} = 0$;
  \item[(iii)] $f$ is flat and unramified.
\end{itemize}

\item Show that a morphism $f : X \to Y$ of schemes of finite type over $k$ is étale if and only if the following condition is satisfied: for each $x \in X$, let $y = f(x)$. Let $\hat{\fO}_x$ and $\hat{\fO}_y$ be the completions of the local rings at $x$ and $y$. Choose fields of representatives (II, 8.25A) $k(x) \subseteq \hat{\fO}_x$ and $k(y) \subseteq \hat{\fO}_y$ so that $k(y) \subseteq k(x)$ via the natural map $\hat{\fO}_y \to \hat{\fO}_x$. Then our condition is that for every $x \in X$, $k(x)$ is a separable algebraic extension of $k(y)$, and the natural map
\begin{equation*}
  \hat{\fO}_y \otimes_{k(y)} k(x) \to \hat{\fO}_x
\end{equation*}
is an isomorphism.

\item If $x$ is a point of a scheme $X$, we define an \textit{étale neighborhood} of $x$ to be an étale morphism $f : U \to X$, together with a point $x' \in U$, such that $f(x') = x$. As an example of the use of étale neighborhoods, prove the following: if $\fF$ is a coherent sheaf on $X$, and if every point of $X$ has an étale neighborhood $f : U \to X$ for which $f^*\fF$ is a free $\fO_U$-module, then $\fF$ is locally free on $X$.

\item Let $Y$ be the plane nodal cubic curve $y^2 = x^2(x + 1)$. Show that $Y$ has a finite étale covering $X$ of degree $2$2, where $X$ is a union of two irreducible components, each one isomorphic to the normalization of $Y$.

\item \textit{(Serre). A linear system with moving singularities.} Let $k$ be an algebraically closed field of characteristic $2$. Let $P_1, \dots, P_7 \in \PP_k^2$ be the seven points of the projective plane over the prime field $\mathbb{F}_2 \subseteq k$. Let $\goth{d}$ be the linear system of all cubic curves in $X$ passing through $P_1, \dots, P_7$.
\begin{itemize}
  \item[(a)] $\goth{d}$ is a linear system of dimension $2$ with base points $P_1, \dots, P_7$, which determines an inseparable morphism of degree $2$ from $X - \{P_i\}$ to $\PP^2$.
  \item[(b)] Every curve $C \in \goth{d}$ is singular. More precisely, either $C$ consists of $3$ lines all passing through one of the $P_i$, or $C$ is an irreducible cuspidal cubic with cusp $P \neq \text{any $P_i$}$. Furthermore, the correspondence $C \mapsto \text{the singular point of $C$}$ is a one-to-one correspondence between $\goth{d}$ and $\PP^2$. Thus, the singular points of elements of $\goth{d}$ move all over.
\end{itemize}

\item \textit{A linear system with moving singularities contained in the base locus (any characteristic).} In affine $3$-space with coordinates $x, y, z$, let $C$ be the conic $(x - 1)^2 + y^2 = 1$ in the $xy$-plane, and let $P$ be the point $(0, 0, t)$ on the $z$-axis. Let $Y_t$ be the closure in $\PP^3$ of the cone over $C$ with vertex $P$. Show that as $t$ varies, the surfaces $\{Y_t \}$ form a linear system of dimension $1$, with a moving singularity at $P$. The base locus of this linear system is the conic $C$ plus the $z$-axis.

\item Let $f : X \to Y$ be a morphism of varieties over $k$. Assume that $Y$ is regular, $X$ is Cohen-Macaulay, and that every fiber of $f$ has dimension equal to $\dim{X} - \dim{Y}$. Then $f$ is flat.

\end{enumerate}
\end{document}
