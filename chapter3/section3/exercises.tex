\documentclass{article}
\usepackage[margin=0.75in]{geometry}
\usepackage{amsmath}
\usepackage{amsthm}
\usepackage{amssymb}
\usepackage{enumitem}
\usepackage{tikz-cd}
\usepackage{yfonts}
\usepackage{mathrsfs}
\usepackage{xcolor}
\DeclareMathAlphabet{\mathpzc}{OT1}{pzc}{m}{it}
\newcommand{\goth}[1]{\mathfrak{#1}}
\newcommand{\fF}{\mathscr{F}}
\newcommand{\fG}{\mathscr{G}}
\newcommand{\fE}{\mathscr{E}}
\newcommand{\fO}{\mathscr{O}}
\newcommand{\fL}{\mathscr{L}}
\newcommand{\fM}{\mathscr{M}}
\newcommand{\fI}{\mathscr{I}}
\newcommand{\fT}{\mathscr{T}}
\newcommand{\fK}{\mathscr{K}}
\newcommand{\fS}{\mathscr{S}}
\newcommand{\fN}{\mathscr{N}}
\newcommand{\fJ}{\mathscr{J}}
\newcommand{\fR}{\mathscr{R}}
\newcommand{\fH}{\mathscr{H}}
\newcommand{\PP}{\mathbb{P}}
\newcommand{\gm}{\goth{m}}
\newcommand{\A}{\mathbb{A}}
\newcommand{\R}{\mathbb{R}}
\newcommand{\C}{\mathbb{C}}
\newcommand{\Q}{\mathbb{Q}}
\newcommand{\N}{\mathbb{N}}
\newcommand{\Z}{\mathbb{Z}}
\newtheorem{theorem}{Theorem}
\newtheorem{lemma}{Lemma}
\newtheorem{corollary}{Corollary}
\DeclareMathOperator{\id}{id}
\DeclareMathOperator{\bProj}{\mathpzc{Proj}}
\DeclareMathOperator{\Frac}{Frac}
\DeclareMathOperator{\rk}{rank}
\DeclareMathOperator{\pic}{Pic}
\DeclareMathOperator{\cacl}{CaCl}
\DeclareMathOperator{\trd}{tr.d.}
\DeclareMathOperator{\cl}{Cl}
\DeclareMathOperator{\depth}{depth}
\DeclareMathOperator{\Div}{Div}
\DeclareMathOperator{\coker}{coker}
\DeclareMathOperator{\len}{length}
\DeclareMathOperator{\height}{height}
\DeclareMathOperator{\supp}{Supp}
\DeclareMathOperator{\proj}{Proj}
\DeclareMathOperator{\im}{im}
\DeclareMathOperator{\Hom}{Hom}
\DeclareMathOperator{\Der}{Der}
\DeclareMathOperator{\spec}{Spec}
\newcommand\srestr[2]{{\left.\kern-\nulldelimiterspace #1\vphantom{\small|} \right|_{#2}}}
\newcommand\restr[2]{{\left.\kern-\nulldelimiterspace #1 \vphantom{\big|} \right|_{#2}}}

\title{Chapter 3, Section 3}
\author{James Lee}
\pagecolor[RGB]{20,20,20}
\color[RGB]{255,255,255}

\begin{document}
\maketitle
\begin{enumerate} [label=\textbf{\arabic*.}, leftmargin=0em]

\item[\textbf{1.}] Let $X$ be a Noetherian scheme.
Show that $X$ is affine if and only if $X_\text{red}$ (II, Ex. 2.3) is affine.

\begin{proof}

\end{proof}

\item[\textbf{2.}] Let $X$ be a reduced Noetherian scheme.
Show that $X$ is affine if and only if each irreducible component is affine.

\begin{proof}
  
\end{proof}

\item[\textbf{6.}] Let $X$ be a Noetherian scheme.
\begin{itemize}
  \item[(a)] Show that the sheaf $\fG$ constructed in the proof of (3.6) is an injective object in the category $\goth{Qco}(X)$ of quasi-coherent sheaves on $X$.
  Thus, $\goth{Qco}(X)$ has enough injectives.

  \item[(b)] Show that any injective object of $\goth{Qco}(X)$ is flasque.

  \item[(c)] Conclude that one can compute cohomology as the derived functors of $\Gamma(X, \cdot)$, considered as a functor $\goth{Qco}(X)$ to $\goth{Ab}$.
\end{itemize}

\begin{proof} $ $ \vspace{0pt}
\begin{itemize} [leftmargin=0cm]
\item[(a)]

\item[(b)]

\item[(c)]
\end{itemize} 
\end{proof}

\item[\textbf{7.}] Let $A$ be a Noetherian ring, let $X = \spec{A}$, let $\goth{a} \subseteq A$ be an ideal, and let $U \subseteq X$ be the open set $X - V(\goth{a})$.
\begin{itemize}
  \item[(a)] For any $A$-module $M$, establish the following formula of Deligne:
  \begin{equation*}
    \Gamma(U, \tilde{M}) \cong \varinjlim_n{\text{Hom}_A(\goth{a}^n, M)}.
  \end{equation*}
  \item[(b)] Apply this in the case of an injective $A$-module $I$, to give another proof of (3.4).
\end{itemize}

\begin{proof} $ $ \vspace{0pt}
\begin{itemize} [leftmargin=0cm]
\item[(a)]

\item[(b)]
\end{itemize} 
\end{proof}

\end{enumerate}
\end{document}

% \item Let $A$ be a Noetherian ring, and let $\goth{a}$ be an ideal of $A$.
% \begin{itemize}
%   \item[(a)] Show that $\Gamma_\goth{a}(\cdot)$ (II, Ex. 5.6) is left-exact functor from the category of $A$-modules to itself.
%   We denote its right derived functors, calculated in $\goth{Mod}(A)$, by $H_\goth{a}^i(\cdot)$.

%   \item[(b)] Now let $X = \spec{A}$, $Y = V(\goth{a})$.
%   Show that for any $A$-module $M$,
%   \begin{equation*}
%     H_\goth{a}^i(M) = H_Y^i(X, \tilde{M}),
%   \end{equation*}
%   where $H_Y^i(X, \cdot)$ denotes cohomology with supports in $Y$ (Ex. 2.3).

%   \item[(c)] For any $i$, show that $\Gamma_\goth{a}(H_\goth{a}^i(M)) = H_\goth{a}^i(M)$.
% \end{itemize}

% \item \textit{Cohomological Interpretation of Depth.} If $A$ is a ring, $\goth{a}$ an ideal, and $M$ an $A$-module, then $\textit{depth}_\goth{a}~M$ is the maximum length of an $M$-regular sequence $x_1, \dots, x_r$, with all $x_i \in \goth{a}$. This generalizes the notion of depth introduced in (II, \S 8).
% \begin{itemize}
%   \item[(a)] Assume that $A$ is Noetherian. Show that if $\text{depth}_\goth{a}~M \geq 1$, then $\Gamma_\goth{a}(M) = 0$, and the converse is true if $M$ is finitely generated.
%   \item[(b)] Show inductively, for $M$ finitely generated, that for any $n \geq 0$, the following conditions are equivalent:
%   \begin{itemize}
%     \item[(i)] $\text{depth}_\goth{a}~M \geq n$;
%     \item[(ii)] $H_\goth{a}^i(M) = 0$ for all $i < n$.
%   \end{itemize}
% \end{itemize}

% \item Let $X$ be a Noetherian scheme, and let $P$ be a closed point of $X$.
% Show that the following conditions are equivalent:
% \begin{itemize}
%   \item[(i)] $\depth \fO_P geq 2$;
%   \item[(ii)] if $U$ is any open neighborhood of $P$, then every section of $\fO_X$ over $U - P$ extends uniquely to a section of $\fO_X$ of $U$.
% \end{itemize}

% This generalizes (I, Ex. 3.20), in view of (II, 8.22A).

% \item Without the Noetherian hypothesis, (3.3) and (3.4) are false.
% Let $A = k[x_0, x_1, x_2, \dots]$ with the relations $x_0^n x_n = 0$ for $n = 1, 2, \dots$.
% Let $I$ be an injective $A$-module containing $A$.
% Show that $I \to I_{x_0}$ is not surjective.
