\documentclass{article}
\usepackage[margin=0.75in]{geometry}
\usepackage{amsmath}
\usepackage{amsthm}
\usepackage{amssymb}
\usepackage{enumitem}
\usepackage{tikz-cd}
\usepackage{yfonts}
\usepackage{mathrsfs}
\usepackage{xcolor}
\DeclareMathAlphabet{\mathpzc}{OT1}{pzc}{m}{it}
\newcommand{\goth}[1]{\mathfrak{#1}}
\newcommand{\fF}{\mathscr{F}}
\newcommand{\fG}{\mathscr{G}}
\newcommand{\fE}{\mathscr{E}}
\newcommand{\fO}{\mathscr{O}}
\newcommand{\fL}{\mathscr{L}}
\newcommand{\fM}{\mathscr{M}}
\newcommand{\fI}{\mathscr{I}}
\newcommand{\fT}{\mathscr{T}}
\newcommand{\fK}{\mathscr{K}}
\newcommand{\fS}{\mathscr{S}}
\newcommand{\fN}{\mathscr{N}}
\newcommand{\fJ}{\mathscr{J}}
\newcommand{\fR}{\mathscr{R}}
\newcommand{\fH}{\mathscr{H}}
\newcommand{\PP}{\mathbb{P}}
\newcommand{\gm}{\goth{m}}
\newcommand{\A}{\mathbb{A}}
\newcommand{\R}{\mathbb{R}}
\newcommand{\C}{\mathbb{C}}
\newcommand{\Q}{\mathbb{Q}}
\newcommand{\N}{\mathbb{N}}
\newcommand{\Z}{\mathbb{Z}}
\newtheorem{theorem}{Theorem}
\newtheorem{lemma}{Lemma}
\newtheorem{corollary}{Corollary}
\DeclareMathOperator{\id}{id}
\DeclareMathOperator{\bProj}{\mathpzc{Proj}}
\DeclareMathOperator{\Frac}{Frac}
\DeclareMathOperator{\rk}{rank}
\DeclareMathOperator{\pic}{Pic}
\DeclareMathOperator{\cacl}{CaCl}
\DeclareMathOperator{\trd}{tr.d.}
\DeclareMathOperator{\cl}{Cl}
\DeclareMathOperator{\Div}{Div}
\DeclareMathOperator{\coker}{coker}
\DeclareMathOperator{\len}{length}
\DeclareMathOperator{\height}{height}
\DeclareMathOperator{\supp}{Supp}
\DeclareMathOperator{\proj}{Proj}
\DeclareMathOperator{\im}{im}
\DeclareMathOperator{\Hom}{Hom}
\DeclareMathOperator{\Der}{Der}
\DeclareMathOperator{\spec}{Spec}
\newcommand\srestr[2]{{\left.\kern-\nulldelimiterspace #1\vphantom{\small|} \right|_{#2}}}
\newcommand\restr[2]{{\left.\kern-\nulldelimiterspace #1 \vphantom{\big|} \right|_{#2}}}

\title{Chapter 3, Section 2}
\author{James Lee}
\pagecolor[RGB]{20,20,20}
\color[RGB]{255,255,255}

\begin{document}
\maketitle
\begin{enumerate} [label=\textbf{\arabic*.}, leftmargin=0em]
\item[\textbf{4.}] \textit{Mayer-Vietoris Sequence.} Let $Y_1, Y_2$ be two closed subsets of $X$. Then there is a long exact sequence of cohomology with supports
\[ \begin{tikzcd}
    \cdots \arrow[r] & {H^i_{Y_1 \cap Y_2}(X, \fF)} \arrow[r]       & {H_{Y_1}^i(X, \fF) \oplus H_{Y_2}^i(X, \fF)} \arrow[r] & {H^i_{Y_1 \cup Y_2}(X, \fF)} \arrow[r] & {} \\
    {} \arrow[r]     & {H^{i + 1}_{Y_1 \cap Y_2}(X, \fF)} \arrow[r] & \cdots .                                               &                                        &   
\end{tikzcd} \] 

\begin{proof}
    There is an exact sequence of sheaves
    \[ \begin{tikzcd}
        & 0 \arrow[d]                                                  & 0 \arrow[d]                        & 0 \arrow[d]                                            &   \\
0 \arrow[r] & \fH^0_{Y_1 \cap Y_2}(\fF) \arrow[r] \arrow[d]                & \fF \arrow[r] \arrow[d]            & \fF_{X - Y_1 \cap Y_2} \arrow[r] \arrow[d]             & 0 \\
0 \arrow[r] & \fH_{Y_1}^0(\fF) \oplus \fH_{Y_2}^0(\fF) \arrow[d] \arrow[r] & \fF \oplus \fF \arrow[d] \arrow[r] & \fF_{X - Y_1} \oplus \fF_{X - Y_2} \arrow[d] \arrow[r] & 0 \\
0 \arrow[r] & \fH^0_{Y_1 \cup Y_2}(\fF) \arrow[d] \arrow[r]                & \fF \arrow[d] \arrow[r]            & \fF_{X - Y_1 \cup Y_2} \arrow[d] \arrow[r]             & 0 \\
        & 0                                                            & 0                                  & 0                                                      &  
\end{tikzcd} \]
which induces the desired long sequence of cohomology with supports by (1.1A).
\end{proof}

\item[\textbf{6.}] Let $X$ be a Noetherian topological space, and let $\{ \fI_\alpha \}_{\alpha \in A}$ be a direct system of injective sheaves of Abelian groups on $X$. Then $\varinjlim \fI _\alpha$ is also injective.

\begin{proof} 
We follow the hint. One direction is clear. Conversely, let $i : \fN \to \fM$ be an injective morphism of sheaves. By the proof of (2.7) we can write $\fN = \varinjlim \fN_\beta$ where $\fN_\beta$ is generated by the sections on some open $U_\beta$, and similarly for $\fM = \varinjlim \fM_\beta$. Notice that we can assume $\fN$ and $\fM$ are defined over the same direct system so that they belong to the same Abelian category. Thus, the inclusion map $i : \fN \to \fM$ can be broken down into inclusion maps $i_\beta : \fN_\beta \to \fM_\beta$. A direct system of morphisms $\fN_\beta \to \fM$ induces the same inclusion morphism $\fN = \varinjlim \fN_\beta \to \fM$, so we reduce to the case when $\fN$ and $\fM$ are generated by a single section over some open set $U$. We have an exact sequence
\[ \begin{tikzcd}
    0 \arrow[rd] & 0 \arrow[d]              &                           &                         &   \\
                 & \fK \arrow[rd] \arrow[d] &                           & 0 \arrow[d]             &   \\
    0 \arrow[r]  & \fR \arrow[r]            & \Z_U \arrow[r] \arrow[rd] & \fN \arrow[r] \arrow[d] & 0 \\
                 &                          &                           & \fM \arrow[rd]          &   \\
                 &                          &                           &                         & 0
    \end{tikzcd} \]
where all the maps are natural, and $\fR, \fK$ are kernels of the quotients $\fN, \fM$, respectively. It is not hard to see from above that any $f : \fN \to \fI$ naturally extends to $\fM$, which is what we wanted to show.

Next, we show any subsheaf $\fR \subseteq \Z_U$ such that $\Z_U / \fR$ is generated by a single section must be finitely generated. Indeed, fix some $x \in X$. Following the proof of (2.7), there exists some open neighborhood $x \in V \subseteq U$ such that $\restr{\fR}{V} \cong d \cdot \restr{\Z}{V}$ for some positive integer $d$. Since $X$ is noetherian, we can cover $U$ by finite number of such $V$, say $V_i$ for $i = 1, \dots, n$. Therefore, there is an exact sequence
\[ \begin{tikzcd}
    0 \arrow[r] & \fR \arrow[r] & \bigoplus_{i = 1}^n d_i \cdot \Z_{V_i} \arrow[r] & {\bigoplus_{i,j,k} d_{ijk} \cdot \Z_{V_i \cap V_j \cap V_k}}
    \end{tikzcd} \]
where $d_{ijk}$ is the minimum of $d_i, d_j, d_k$. The terms on the right are finitely generated. Thus, $\fR$ is finitely generated, and any $\fR \to \varinjlim \fI_\alpha$ must factor through one of the $\fI_\alpha$ (each generator $s_i$ of $\fR$ factors through one of the $\fI_{\alpha_i}$, so take any $\beta > \alpha_i$, which exists by definition of a direct system).
\end{proof}

\item[\textbf{7.}] Let $S^1$ be the circle (with its usual topology), and let $\Z$ be the constant sheaf $\Z$.
\begin{itemize}
    \item[(a)] Show that $H^1(S^1, \Z) = \Z$, using our definition of cohomology.
    \item[(b)] Now let $\fR$ be the sheaf of germs of continuous real-valued functions on $S^1$. Show that $H^1(S^1, \fR) = 0$.
\end{itemize}

\begin{proof} $ $ \vspace{0pt}
\begin{itemize} [leftmargin=0cm]
\item[(a)] We remark that cohomology commutes with colimits on paracompact Hausdorff spaces. In particular, the statements of (II, Ex.1.11), (2.9) hold for $S^1$. Let $A, B$ be closed subsets of $S^1$ homeomorphic to the unit interval such that $A \cup B = S^1$ and $A \cap B = \{ P, Q \}$ for two distinct points $P, Q$ in $S^1$ (in the obvious way...). From now on, for any closed subset $C$ of $S^1$, denote $\Z_C = i_* \Z$, where $i : C \hookrightarrow S^1$ is the inclusion map and $\Z$ is the constant sheaf on $C$. Without ambiguity $\Z$ will denote the constant sheaf on the ambient space. We claim the following sequence of sheaves
\[\begin{tikzcd}
    0 \arrow[r] & \Z \arrow[r, "\Delta"] & \Z_A \oplus \Z_B \arrow[r, "\tau"] & \Z_{A \cap B} \arrow[r] & 0
    \end{tikzcd}\]
defined by $\Delta(a) = (a, a)$ and $\tau(a, b) = b - a$ is exact. In particular, there exists natural maps $i^\#_A, i^\#_B : \Z \to \Z_A, \Z_B$ associated to the inclusion maps $i_A, i_B : A, B \hookrightarrow S^1$, so that $\Delta = (i_A^\#, i_B^\#)$. In the same way, the associated morphism of sheaves of the inclusion maps $j_A, j_B : A \cap B \to A, B$ ascends to naturally defined maps $j_A^\#, j_B^\# : \Z_A, \Z_B \to \Z_{A \cap B}$ in the form of a restriction morphism. Thus, $\tau = j_B^\# - j_A^\#$. Exactness can be checked at the level of stalks. Suppose $R \notin A \cap B$. Then either $R \in A$ or $R \in B$, say $R \in A$. Then the stalks are $(\Z)_R = \Z$, $(\Z_A)_R = \Z$, $(\Z_B)_R = 0$, which is an exact sequence. If $R \in A \cap B$, then the stalks are $(\Z)_R = \Z$, $(\Z_A)_R = \Z$, $(\Z_B)_R = \Z$, $\Z_{A \cap B} = \Z$ defined by $\Delta$ and $\tau$, which is clearly exact. Hence, the sequence is exact at all points, so the sequence is exact.

Taking cohomology, we get a long exact sequence of cohomology groups
\[ \begin{tikzcd}
    0 \arrow[r]  & H^0(S^1, \Z) \arrow[r] & H^0(S^1, \Z_A) \oplus H^0(S^1, \Z_B) \arrow[r, "\tau_0"] & H^0(S^1, \Z_{A \cap B}) \arrow[r] & {}     \\
    {} \arrow[r] & H^1(S^1, \Z) \arrow[r] & H^1(S^1, \Z_A) \oplus H^1(S^1, \Z_B) \arrow[r] & H^1(S^1, \Z_{A \cap B}) \arrow[r] & \cdots
    \end{tikzcd} \]
where $H^i(\Z_A \oplus \Z_B) \cong H^i(\Z_A) \oplus H^i(\Z_B)$ by (2.9). By (2.10), we have
\begin{align*}
    H^0(S^1, \Z), H^0(S^1, \Z_A), H^0(S^1, \Z_B) & = \Z, \\
    H^0(\Z_{A \cap B}) & = \Z \oplus \Z, \\
    H^1(\Z_{A \cap B}) & = 0
\end{align*}
The first line follows from the fact that $A, B , S^1$ are all connected and locally connected. The intersection $A \cap B$ is a noetherian space of dimension zero with two irreducible components, namely the points $P$ and $Q$, so its space of global sections is free of rank two. Lastly, $H^1(S^1, \Z_{A \cap B}) = H^1(A \cap B, \Z) = 0$ by (2.7). By exactness, we reduce to the following exact sequence
\[ \begin{tikzcd}
    0 \arrow[r] & \Z \oplus \Z / \im{\tau_0} \arrow[r] & H^1(S^1,\Z) \arrow[r] & H^1(S^1, \Z_A) \oplus H^1(S^1, \Z_B).
    \end{tikzcd} \]
The homomorphism $\tau_0$ is defined by $\tau_0(a, b) = (b - a, b - a)$, which is the diagonal map. Thus, the term on the left is free of rank one. It remains to show $H^1(S^1, \Z_A) = H^1(S^1, \Z_B) = 0$. By (2.10), it suffices to show $H^1(A, \Z) = 0$.

From here, $\Z$ will denote the constant sheaf on $A$. Identifying $A$ with the closed unit interval $[0, 1]$, we repeat the procedure above for $A$. Pick any $t \in (0, 1)$, say $t = 2^{-1}$. Then $X = [0, t]$ and $Y = [t, 1]$ cover $A$, so taking cohomology groups, we get a long exact sequence
\[ \begin{tikzcd}
    0 \arrow[r]  & H^0(A, \Z) \arrow[r] & H^0(A, \Z_X) \oplus H^0(A, \Z_Y) \arrow[r] & H^0(A, \Z_{X \cap Y}) \arrow[r] & {}     \\
    {} \arrow[r] & H^1(A, \Z) \arrow[r] & H^1(A, \Z_X) \oplus H^1(A, \Z_Y) \arrow[r] & H^1(A, \Z_{X \cap Y}) \arrow[r] & \cdots
    \end{tikzcd} \]
Imitating the previous calculation, the first row is exact and $H^1(A, \Z_{X \cap Y}) = 0$ by (2.7) and (2.10). Thus, we reduce to the following exact sequence
\[ \begin{tikzcd}
    0 \arrow[r] & H^1(A, \Z) \arrow[r] & H^1(A,\Z_X) \oplus H^1(A, \Z_Y) \arrow[r] & 0.
    \end{tikzcd} \]
Since $X \cong A, B$, $H^1(A, \Z_X), H^1(A, \Z_Y) \cong H^1(A, \Z)$ by (2.10), which is possible if and only if $H^1(A, \Z) = 0$.

\item[(b)] Let $\fM$ be the sheaf of germs of measurable real-valued functions on $S^1$ modulo equivalence almost everywhere. It is clearly flasque, since for any measurable $f : V \to \R$ where $V \subseteq U \subseteq \R$ are open sets, the extension of $f$ by zero on $U$ is a measurable function. Thus, we have an exact sequence of sheaves
\[ \begin{tikzcd}
    0 \arrow[r] & \fR \arrow[r] & \fM \arrow[r] & \fM/\fR \arrow[r] & 0.
    \end{tikzcd} \]
Any measurable function on $S^1$ is continuous except for on a set of measure zero in $S^1$, so any global section of $\fM / \fR$ is locally zero almost everywhere. Since we are considering equivalence classes of functions, where two functions are considered equal if they are equal except for on a measure zero set, we conclude that $\fM / \fR$ has no non-zero global section. Taking cohomology groups, we have $H^1(S^1, \fR) = 0$.

\end{itemize}
\end{proof}

\end{enumerate}
\end{document}

% \item \begin{itemize}
%     \item[(a)] Let $X = \A_k^1$ be the affine line over an infinite field $k$. Let $P, Q$ be distinct closed points of $X$, and let $U = X - \{ P, Q \}$. Show that $H^1(X, \Z_U) \neq 0$.
%     \item[(b)] More generally, let $Y \subseteq X = \A_k^n$ be the union of $n + 1$ hyperplanes in suitable general positions, and let $U = X - Y$. Show that $H^n(X, \Z_U) \neq 0$. Thus, the result of (2.7) is the best possible.
% \end{itemize}

% \item Let $X = \PP_k^1$ be the projective line over an algebraically closed field $k$. Show that the exact sequence $0 \to \fO \to \fK \to \fK / \fO \to 0$ of (II, Ex.1.21d) is a flasque resolution of $\fO$. Conclude from (II, Ex. 1.21e) that $H^i(X, \fO) = 0$ for all $i > 0$.

% \item \textit{Cohomology with Supports.} Let $X$ be a topological space, let $Y$ be a closed subset, and let $\fF$ be a sheaf of abelian groups. Let $\Gamma_Y(X, \fF)$ denote the group of sections of $\fF$ with support in $Y$ (II, Ex. 1.20).
% \begin{itemize}
%     \item[(a)] Show that $\Gamma_Y(X, \cdot)$ is a left exact functor from $\goth{Ab}(X)$ to $\goth{Ab}$.

%     We denote the right derived functors of $\Gamma_Y(X, \cdot)$ by $H_Y^i(X, \cdot)$. They are the \textit{cohomology groups of $X$ with supports in $Y$,} and coefficients in a given sheaf.

%     \item[(b)] If $0 \to \fF' \to \fF \to \fF'' \to 0$ is an exact sequence of sheaves, with $\fF'$ flasque, show that
%     \begin{equation*}
%         0 \to \Gamma_Y(X, \fF') \to \Gamma_Y(X, \fF) \to \Gamma_Y(X, \fF'') \to 0
%     \end{equation*}
%     is exact.

%     \item[(c)] Show that if $\fF$ is flasque, then $H^i_Y(X, \fF) = 0$ for all $i > 0$.

%     \item[(d)] If $\fF$ is flasque, show that the sequence
%     \begin{equation*}
%         0 \to \Gamma_Y(X, \fF) \to \Gamma(X, \fF) \to \Gamma(X - Y, \fF) \to 0
%     \end{equation*}
%     is exact.

%     \item[(e)] Let $U = X - Y$. Show that for any $\fF$, there is a long exact sequence of cohomology groups
%     \[ \begin{tikzcd}
%         0 \arrow[r]  & {H_Y^0(X, \fF)} \arrow[r] & {H^0(X, \fF)} \arrow[r] & {H^0(U, \restr{\fF}{U})} \arrow[r] & {} \\
%         {} \arrow[r] & {H_Y^1(X, \fF)} \arrow[r] & {H^1(X, \fF)} \arrow[r] & {H^1(U, \restr{\fF}{U})} \arrow[r] & {} \\
%         {} \arrow[r] & {H_Y^2(X, \fF)} \arrow[r] & \cdots .                  &                                     &   
%         \end{tikzcd} \]
    
%     \item[(f)] \textit{Excision.} Let $V$ be an open subset of $X$ containing $Y$. Then there are natural functorial isomorphisms, for all $i$ and $\fF$,
%     \begin{equation*}
%         H_Y^i(X, \fF) \cong H_Y^i(V, \restr{\fF}{V}).
%     \end{equation*}
% \end{itemize}

% \item Let $X$ be a Zariski space (II, Ex. 3.17). Let $P \in X$ be a closed point, and let $X_P$ be the subset of $X$ containing of all points $Q \in X$ such that $P \in \overline{\{Q \}}$. We call $X_P$ the \textit{local space} of $X$ at $P$, and give it the induced topology. Let $j : X_P \to X$ be the inclusion, and for any sheaf $\fF$ on $X$, let $\fF_P = j^* \fF$ Show that for all $i$, $\fF$, we have
% \begin{equation*}
%     H_P^i(X, \fF) = H_P^i(X_P, \fF_P).
% \end{equation*}