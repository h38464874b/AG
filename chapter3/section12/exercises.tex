\documentclass{article}
\usepackage[margin=0.75in]{geometry}
\usepackage{amsmath}
\usepackage{amsthm}
\usepackage{amssymb}
\usepackage{enumitem}
\usepackage{tikz-cd}
\usepackage{yfonts}
\usepackage{mathrsfs}
\DeclareMathAlphabet{\mathpzc}{OT1}{pzc}{m}{it}
\newcommand{\goth}[1]{\mathfrak{#1}}
\newcommand{\fF}{\mathscr{F}}
\newcommand{\fG}{\mathscr{G}}
\newcommand{\fE}{\mathscr{E}}
\newcommand{\fO}{\mathscr{O}}
\newcommand{\fL}{\mathscr{L}}
\newcommand{\fM}{\mathscr{M}}
\newcommand{\fI}{\mathscr{I}}
\newcommand{\fT}{\mathscr{T}}
\newcommand{\fK}{\mathscr{K}}
\newcommand{\fS}{\mathscr{S}}
\newcommand{\fJ}{\mathscr{J}}
\newcommand{\fR}{\mathscr{R}}
\newcommand{\fH}{\mathscr{H}}
\newcommand{\fN}{\mathscr{N}}
\newcommand{\PP}{\mathbb{P}}
\newcommand{\gm}{\goth{m}}
\newcommand{\A}{\mathbb{A}}
\newcommand{\R}{\mathbb{R}}
\newcommand{\C}{\mathbb{C}}
\newcommand{\Q}{\mathbb{Q}}
\newcommand{\N}{\mathbb{N}}
\newcommand{\Z}{\mathbb{Z}}
\newtheorem{theorem}{Theorem}
\newtheorem{lemma}{Lemma}
\newtheorem{corollary}{Corollary}
\DeclareMathOperator{\id}{id}
\DeclareMathOperator{\bProj}{\mathpzc{Proj}}
\DeclareMathOperator{\Frac}{Frac}
\DeclareMathOperator{\rk}{rank}
\DeclareMathOperator{\pic}{Pic}
\DeclareMathOperator{\cacl}{CaCl}
\DeclareMathOperator{\trd}{tr.d.}
\DeclareMathOperator{\cl}{Cl}
\DeclareMathOperator{\Div}{Div}
\DeclareMathOperator{\coker}{coker}
\DeclareMathOperator{\len}{length}
\DeclareMathOperator{\height}{height}
\DeclareMathOperator{\supp}{Supp}
\DeclareMathOperator{\proj}{Proj}
\DeclareMathOperator{\im}{im}
\DeclareMathOperator{\Hom}{Hom}
\DeclareMathOperator{\Der}{Der}
\DeclareMathOperator{\spec}{Spec}
\DeclareMathOperator{\rHom}{\mathpzc{Hom}}
\newcommand\srestr[2]{{
  \left.\kern-\nulldelimiterspace % automatically resize the bar with \right
  #1 % the function
  \vphantom{\small|} % pretend it's a little taller at normal size
  \right|_{#2} % this is the delimiter
}}
\newcommand\restr[2]{{% we make the whole thing an ordinary symbol
  \left.\kern-\nulldelimiterspace % automatically resize the bar with \right
  #1 % the function
  \vphantom{\big|} % pretend it's a little taller at normal size
  \right|_{#2} % this is the delimiter
}}

% 2.4, 2.6, 2.7
% 3.1, 3.2, 3.6, 3.7
% 4.1, 4.2, 4.3, 4.4, 4.5
% 5.1, 5.2, 5.3, 5.10
% 6.1, 6.3, 6.6, 6.7
% 7.1, 7.3
% 8.1, 8.2, 8.3, 
% 9.3, 9.4, 9.11
% 10.1, 10.2, 10.3, 10.5, 10.6
% 11.1, 11.2, 11.8
% 12.1, 12.2

\title{Chapter 3, Section 12}

\usepackage{xcolor}

\pagecolor[RGB]{8,27,31}

\color[RGB]{255,255,255}

\begin{document}
\maketitle
\begin{enumerate} [label=\textbf{\arabic*.}, leftmargin=0em]

\item Let $Y$ be a scheme of finite type over an algebraically closed field $k$. Show that the function
\begin{equation*}
  \varphi(y) = \dim_k(\goth{m}_y/\goth{m}_y^2)
\end{equation*}
is upper semicontinuous on the set of closed points of $Y$.

\item Let $\{X_t \}$ be a family of hypersurfaces of the same degree in $\PP_k^n$. Show that for each $i$, the function $h^i(X_t, \fO_{X_t})$ is a constant function of $t$.

\item Let $X_1 \subseteq \PP_k^4$ be the \textit{rational normal quartic curve} (which is the $4$-uple embedding of $\PP^1$ in $\PP^4$). Let $X_0 \subseteq \PP_k^3$ be a nonsingular rational quartic curve, such as the one in (I, Ex. 3.18b). Use (9.8.3) to construct a flat family $\{X_t\}$ of curves in $\PP^4$, parameterized by $T = \A^1$, with the given fibers $X_1$ and $X_0$ for $t = 1$ and $t = 0$.

Let $\fI \subseteq \fO_{\PP^4 \times T}$ be the ideal sheaf of the total family $X \subseteq \PP^4 \times T$. Show that $\fI$ is flat over $T$. Then show that
\begin{equation*}
  h^0(t, \fI) = \begin{cases}
    0 & \text{for $t \neq 0$} \\
    1 & \text{for $t = 0$}
  \end{cases}
\end{equation*}
and also
\begin{equation*}
  h^1(t, \fI) = \begin{cases}
    0 & \text{for $t \neq 0$} \\
    1 & \text{for $t = 0$.}
  \end{cases}
\end{equation*}
This gives another example of cohomology groups jumping at a special point.

\item Let $Y$ be an integral scheme of finite type over an algebraically closed field $k$. Let $f : X \to Y$ be a flat projective morphism whose fibers are all integral schemes. Let $\fL, \fM$ be invertible sheaves on $X$, and assume for each $y \in Y$ that $\fL_y \cong \fM_y$ on the fiber $X_y$. Then show that there is an invertible sheaf $\fN$ on $Y$ such that $\fL \cong \fM \otimes f^* \fN$.

\item Let $Y$ be an integral scheme of finite type over an algebraically closed field $k$. Let $\mathscr{E}$ be a locally free sheaf on $Y$, and let $X = \PP(\fE)$. Then show that $\pic{X} \cong (\pic{Y}) \times \Z$. This strengthens (II, Ex. 7.9).

\item Let $X$ be an integral projective scheme over an algebraically closed field $k$, and assume that $H^1(X, \fO_X) = 0$. Let $T$ be a connected scheme of finite type over $k$.
\begin{itemize}
  \item[(a)] If $\fL$ is an invertible sheaf on $X \times T$, show that the invertible sheaves $\fL_t$ on $X = X \times \{ t \}$ are isomorphic, for all closed points $t \in T$.
  \item[(b)] Show that $\pic(X \times T) = \pic{X} \times \pic{T}$.
\end{itemize}

\end{enumerate}

\end{document}
