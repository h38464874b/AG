\documentclass{article}
\usepackage[margin=0.75in]{geometry}
\usepackage{amsmath}
\usepackage{amsthm}
\usepackage{amssymb}
\usepackage{enumitem}
\usepackage{tikz-cd}
\usepackage{yfonts}
\usepackage{mathrsfs}
\DeclareMathAlphabet{\mathpzc}{OT1}{pzc}{m}{it}
\newcommand{\goth}[1]{\mathfrak{#1}}
\newcommand{\fF}{\mathscr{F}}
\newcommand{\fG}{\mathscr{G}}
\newcommand{\fE}{\mathscr{E}}
\newcommand{\fO}{\mathscr{O}}
\newcommand{\fL}{\mathscr{L}}
\newcommand{\fM}{\mathscr{M}}
\newcommand{\fI}{\mathscr{I}}
\newcommand{\fT}{\mathscr{T}}
\newcommand{\fK}{\mathscr{K}}
\newcommand{\fS}{\mathscr{S}}
\newcommand{\fJ}{\mathscr{J}}
\newcommand{\fR}{\mathscr{R}}
\newcommand{\fH}{\mathscr{H}}
\newcommand{\fN}{\mathscr{N}}
\newcommand{\PP}{\mathbb{P}}
\newcommand{\gm}{\goth{m}}
\newcommand{\A}{\mathbb{A}}
\newcommand{\R}{\mathbb{R}}
\newcommand{\C}{\mathbb{C}}
\newcommand{\Q}{\mathbb{Q}}
\newcommand{\N}{\mathbb{N}}
\newcommand{\Z}{\mathbb{Z}}
\newtheorem{theorem}{Theorem}
\newtheorem{lemma}{Lemma}
\newtheorem{corollary}{Corollary}
\DeclareMathOperator{\id}{id}
\DeclareMathOperator{\bProj}{\mathpzc{Proj}}
\DeclareMathOperator{\Frac}{Frac}
\DeclareMathOperator{\rk}{rank}
\DeclareMathOperator{\pic}{Pic}
\DeclareMathOperator{\cacl}{CaCl}
\DeclareMathOperator{\trd}{tr.d.}
\DeclareMathOperator{\cl}{Cl}
\DeclareMathOperator{\Div}{Div}
\DeclareMathOperator{\coker}{coker}
\DeclareMathOperator{\len}{length}
\DeclareMathOperator{\height}{height}
\DeclareMathOperator{\supp}{Supp}
\DeclareMathOperator{\proj}{Proj}
\DeclareMathOperator{\im}{im}
\DeclareMathOperator{\Hom}{Hom}
\DeclareMathOperator{\Der}{Der}
\DeclareMathOperator{\spec}{Spec}
\DeclareMathOperator{\rHom}{\mathpzc{Hom}}
\newcommand\srestr[2]{{
  \left.\kern-\nulldelimiterspace % automatically resize the bar with \right
  #1 % the function
  \vphantom{\small|} % pretend it's a little taller at normal size
  \right|_{#2} % this is the delimiter
}}
\newcommand\restr[2]{{% we make the whole thing an ordinary symbol
  \left.\kern-\nulldelimiterspace % automatically resize the bar with \right
  #1 % the function
  \vphantom{\big|} % pretend it's a little taller at normal size
  \right|_{#2} % this is the delimiter
}}

% 2.4, 2.6, 2.7
% 3.1, 3.2, 3.6, 3.7
% 4.1, 4.2, 4.3, 4.4, 4.5
% 5.1, 5.2, 5.3, 5.10
% 6.1, 6.3, 6.6, 6.7
% 7.1, 7.3
% 8.1, 8.2, 8.3, 
% 9.3, 9.4, 9.11
% 10.1, 10.2, 10.3, 10.5, 10.6
% 11.1, 11.2, 11.8
% 12.1, 12.2

\title{Chapter 3, Section 9}

\usepackage{xcolor}

\pagecolor[RGB]{8,27,31}

\color[RGB]{255,255,255}

\begin{document}
\maketitle
\begin{enumerate} [label=\textbf{\arabic*.}, leftmargin=0em]

\item A flat morphism $f : X \to Y$ of finite type of Noetherian schemes is open, i.e., for every open subset $U \subseteq X$, $f(U)$ is open in $Y$.

\item Do the calculation of (9.8.4) for the curves of (I, Ex. 3.14). Show that you get an embedded point at the cusp of the plane cubic curve.

\item Some examples of flatness and nonflatness.
\begin{itemize}
  \item[(a)] If $f : X \to Y$ is a finite surjective morphism of nonsingular varieties over an algebraically closed field $k$, then $f$ is flat.
  \item[(b)] Let $X$ be a union of two planes meeting at a point, each of which maps isomorphically to a plane $Y$. Show that $f$ is not flat. For example, let $Y = \spec{k[x, y]}$ and $X = \spec{k[x, y, z, w]} / (z, w) \cap (x + z, y + w)$.
  \item[(c)] Again let $Y = \spec{k[x, y]}$, but take $X = \spec{k[x, y, z, w] / (z^2, zw, w^2, xz - yz)}$. Show that $X_\text{red} \cong Y$, $X$ has no embedded points, but that $f$ is not flat.
\end{itemize}

\item \textit{Open Nature of Flatness.} Let $f : X \to Y$ be a morphism of finite type of Noetherian schemes. Then $\{ x \in X \mid \text{$f$ is flat at $x$} \}$ is an open subset of $X$ (possible empty).

\item \textit{Very Flat Families.} For any closed subscheme $X \subseteq \PP^n$, we denote by $C(X) \subseteq \PP^{n + 1}$ the projective cone over $X$ (I, Ex. 2.10). If $I = \subseteq k[x_0, \dots, x_n]$ is the (largest) homogenous ideal of $X$, then $C(X)$ is defined by the ideal generated by $I$ in $k[x_0, \dots, x_{n + 1}]$.
\begin{itemize}
  \item[(a)] Give an example to show that if $\{ X_t \}$ is a flat family of closed subschemes of $\PP^n$, then $\{ C(X_t) \}$ need not be a flat family in $\PP^{n + 1}$.
  \item[(b)] To remedy this situation, we make the following definition. Let $X \subseteq \PP_T^n$ be a closed subscheme, where $T$ is a Noetherian integral scheme. For each $t \in T$, let $I_t \subseteq S_t = k(t)[x_0, \dots, x_n]$ be the homogenous ideal of $X_t$ in $\PP_{k(t)}^n$. We say that the family $\{X_t \}$ is \textit{very flat} if for all $d \geq 0$,
  \begin{equation*}
    \dim_{k(t)}(S_t / I_t)_d
  \end{equation*}
  is independent of $t$.
  \item[(c)] If $\{X_t\}$ is a very flat family in $\PP^n$, show that it is flat. Show also that $\{ C(X_t) \}$ is a very flat family in $\PP^{n + 1}$, and hence flat.
  \item[(d)] If $\{ X_{(t)}\}$ is an algebraic family of projectively normal varieties in $\PP_k^n$, parametrized by a nonsingular curve $T$ over an algebraically closed field $k$, then $\{X_{(t)} \}$ is a very flat family of schemes.
\end{itemize}

\item Let $Y \subseteq \PP^n$ be a nonsingular variety of dimension $\geq 2$ over an algebraically closed field $k$. Suppose $\PP^{n - 1}$ is a hyperplane in $\PP^n$ which does not contain $Y$, and such that the scheme $Y' = Y \cap \PP^{n - 1}$ is also nonsingular. Prove that $Y$ is a complete intersection in $\PP^n$ if and only if $Y'$ is a complete intersection in $\PP^{n - 1}$.

\item Let $Y \subseteq X$ be a closed subscheme, where $X$ is a scheme of finite type over a field $k$. Let $D = k[t] / (t^2)$ be the ring of dual numbers, and define an \textit{infinitesimal deformation} of $Y$ \textit{as a closed subscheme of $X$}, to be a closed subscheme $Y' \subseteq X \times_k D$, which is flat over $D$, and whose closed fiber is $Y$. Show that these $Y'$ are classified by $H^0(Y, \fN_{Y/X})$, where
\begin{equation*}
  \fN_{Y/X} = \rHom_{\fO_Y}(\mathscr{J}_y/\mathscr{J}_Y^2, \fO_Y).
\end{equation*}

\item Let $A$ be a finitely generated $k$-algebra. Write $A$ as a quotient of a polynomial ring $P$ over $k$, and let $J$ be the kernel:
\begin{equation*}
  0 \to J \to P \to A \to 0.
\end{equation*}
Consider the exact sequence of (II, 8.4A)
\begin{equation*}
  J / J^2 \to \Omega_{P/k} \otimes_P A \to \Omega_{A / k} \to 0.
\end{equation*}
Apply the functor $\text{Hom}_A(\cdot, A)$ and let $T^1(A)$ be the cokernel:
\begin{equation*}
  \Hom_A(\Omega_{P/k} \otimes A, A) \to \Hom_A(J/J^2, A) \to T^1(A) \to 0.
\end{equation*}
Now use the construction of (II, Ex. 8.6) to show that $T^1(A)$ classifies infinitesimal deformations of $A$, i.e., algebras $A'$ flat over $D = k[t] / (t^2)$, with $A' \otimes_D k \cong A$. It follows that $T^1(A)$ is independent of the given representation of $A$ as quotient of a polynomial ring $P$.

\item A $k$-algebra $A$ is said to be \textit{rigid} if it has no infinitesimal deformations, or equivalently, by (Ex. 9.8) if $T^1(A) = 0$. Let $A = k[x, y, z, w] / (x, y) \cap (z, w)$, and show that $A$ is rigid. This corresponds to two planes in $\A^4$ which meet at a point.

\item A scheme $X_0$ over a field $k$ is \textit{rigid} if it has no infinitesimal deformations.
\begin{itemize}
  \item[(a)] Show that $\PP_k^1$ is rigid, using (9.13.2).
  \item[(b)] one might think that if $X_0$ is rigid over $k$, then every global deformation of $X_0$ is locally trivial. Show that this is not so, by constructing a proper, flat morphism $f : X \to \A^2$ over $k$ algebraically closed, such that $X_0 \cong \PP_k^1$, but there is no open neighborhood $U$ of $0$ in $\A^2$ for which $f^{-1}(U) \cong U \times \PP^1$.
  \item[(c)] Show, however, that one can trivialize a global deformation of $\PP^1$ after a flat base extension, in the following sense: let $f : X \to T$ be a flat projective morphism, where $T$ is a nonsingular curve over $k$ algebraically closed. Assume there is a closed point $t \in T$ such that $X_t \cong \PP_k^1$. Then there exists a nonsingular curve $T'$, and a flat morphism $g : T' \to T$, whose image contains $t$, such that if $X' = X \times_T T'$ is the base extension, then the new family $f' : X' \to T'$ is isomorphic to $\PP_T^1 \to T'$.
\end{itemize}

\item Let $Y$ be a nonsingular curve of degree $d$ in $\PP_k^n$, over an algebraically closed field $k$. Show that
\begin{equation*}
  0 \leq p_a(Y) \leq \frac{1}{2}(d - 1)(d - 2).
\end{equation*}


\end{enumerate}
\end{document}
